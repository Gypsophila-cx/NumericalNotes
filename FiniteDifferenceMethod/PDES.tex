\documentclass[a4paper,10pt]{ctexart}
%引用设置使用Bibtex
\usepackage{gbt7714}
\bibliographystyle{gbt7714-numerical}
%页面设置
\usepackage{geometry}
%字体设置
\usepackage{fontspec}
%\setmainfont{Times New Roman}
%定理环境
\usepackage{amsmath}
\numberwithin{equation}{section}
\usepackage{amsthm}
\newtheorem*{definition}{Definition}
\newtheorem{theorem}{Theorem}
\newtheorem{lemma}{Lemma}
\newtheorem*{corollary}{Corollary}
\newtheorem*{proposition}{Proposition}
\newtheorem*{example}{Example}
%数学环境字体
\usepackage{bm}
\usepackage[all]{xy}
%加载 TikZ 用于绘制交换图
\usepackage{tikz-cd}
\usepackage{tikz}
\usepackage{pgfplots}
\newcommand{\tikzdef}{\pgfmathsetmacro} % 在tikzpicture内的foreach循环中定义实数临时变量
%颜色
\usepackage{color,xcolor}

\definecolor{miku}{RGB}{57,197,187}
\definecolor{sakura}{RGB}{255,192,203}
\definecolor{rose}{RGB}{255,228,225}
\definecolor{brown}{RGB}{210,105,30}
\definecolor{lbrown}{RGB}{239,235,224}
\definecolor{bule}{RGB}{0,47,167}
\definecolor{lyellow}{RGB}{250,250,210}
\definecolor{lpurple}{RGB}{255,240,245}
\definecolor{lbule}{RGB}{135,206,250}
\definecolor{gbule}{RGB}{64,224,208}
\definecolor{green}{RGB}{138,200,207}
\definecolor{lgreen}{RGB}{225,255,255}
\definecolor{lorange}{RGB}{248,172,140}
\definecolor{salmon}{RGB}{250,128,114}
\definecolor{burgundy}{rgb}{0.5, 0.0, 0.13}
%链接设置
\usepackage[colorlinks=true,pdfstartview=FitH,linkcolor=blue,anchorcolor=violet, citecolor=magenta]{hyperref} 
%封面
\usepackage{pdfpages}
\usepackage{mathrsfs}
\usepackage{amssymb}
\usepackage{graphicx}
\usepackage{lipsum}
%彩色框
\usepackage{framed}
\usepackage{tcolorbox}
\tcbuselibrary{breakable}
\tcbuselibrary{theorems}
\tcbuselibrary{skins}
\usepackage{colortbl}
\usepackage{float}
\usepackage[export]{adjustbox}
\newtcolorbox[auto counter,number within=section]{notebox}[2][]{%
colback=miku!2!white,
colframe=miku,
coltitle=white,
fonttitle=\bfseries,
rightrule=2pt,
leftrule=2pt,
bottomrule=2pt,
colbacktitle=miku,
theorem style=standard,
breakable,
arc=2pt,
drop fuzzy shadow=black!20!white,
title=Note~\thetcbcounter: #2,#1}
\newtcolorbox[auto counter,number within=section]{markbox}[2][]{%
colback=miku!2!white,
colframe=miku,
coltitle=white,
fonttitle=\bfseries,
rightrule=0pt,
leftrule=0pt,
bottomrule=2pt,
colbacktitle=miku,
theorem style=standard,
breakable,
arc=0pt,
drop fuzzy shadow=black!20!white,
title=Remark~\thetcbcounter: #2,#1}
\newtcolorbox[no counter]{theorems}[2][]{%
width=12cm,
center,
sidebyside,
sidebyside adapt=left,
sidebyside gap=6mm,
sidebyside align=center seam,
colback=burgundy!2!white,
colframe=burgundy,
coltitle=white,
fonttitle=\bfseries,
rightrule=1pt,
leftrule=1pt,
bottomrule=2pt,
colbacktitle=burgundy,
theorem style=standard,
enhanced,
drop fuzzy shadow southeast=black!30!white,
breakable,
arc=0pt,
title=Theorem. #2,#1}
\newtcolorbox[no counter]{definitions}[2][]{%
width=12cm,
center,
colback=lyellow!2!white,
colframe=yellow!3!lyellow,
coltitle=bule,
fonttitle=\bfseries,
rightrule=0pt,
leftrule=1pt,
bottomrule=2pt,
colbacktitle=lyellow,
theorem style=standard,
breakable,
arc=5pt,
enhanced,
drop fuzzy shadow southeast=black!20!white,
title=Definition. #2,#1}
\newtcolorbox[auto counter,number within=section]{corollarys}[2][]{%
colback=lyellow!2!white,
colframe=lyellow,
coltitle=bule,
fonttitle=\bfseries,
rightrule=0pt,
leftrule=1pt,
bottomrule=2pt,
colbacktitle=lyellow,
theorem style=standard,
breakable,
arc=0pt,
enhanced,
drop fuzzy shadow southeast=black!20!white,
title=Corollary~\thetcbcounter: #2,#1}
\newtcolorbox[auto counter,number within=section]{lemmas}[2][]{%
width=12cm,
center,
colback=lyellow!2!white,
colframe=lorange!30!sakura,
coltitle=bule,
fonttitle=\bfseries,
rightrule=0pt,
leftrule=1pt,
bottomrule=2pt,
colbacktitle=lorange!30!sakura,
theorem style=standard,
breakable,
arc=5pt,
enhanced,
drop fuzzy shadow southeast=black!20!white,
title=Lemma. #2,#1}
\newtcolorbox[auto counter,number within=section]{propositions}[2][]{%
width=12cm,
center,
colback=salmon!5,
colframe=salmon!90!black,
coltitle=white,
fonttitle=\bfseries,
rightrule=1pt,
leftrule=1pt,
bottomrule=2pt,
colbacktitle=salmon!90!black,
theorem style=standard,
breakable,
arc=5pt,
enhanced,
drop fuzzy shadow southeast=black!20!white,
title=Proposition. #2,#1}
\newtcolorbox[no counter]{egbox}[2][]{%
width=12cm,
center,
colback=black!5!white,
colframe=black!20!white,
coltitle=black,
fonttitle=\bfseries,
rightrule=1pt,
leftrule=1pt,
bottomrule=2pt,
colbacktitle=black!20!white,
theorem style=standard,
breakable,
arc=0pt,
enhanced,
drop fuzzy shadow southeast=black!20!white,
title=Example. #2,#1}

%\begin{figure}[H]
%\centering
%\includegraphics[center]{pic.png}
%\end{figure}
\geometry{left=3cm,right=3cm,top=2cm,bottom=2cm}
\tcbuselibrary{most}

\usepackage[linesnumbered,ruled,vlined]{algorithm2e}
\usepackage{algorithmic}

\SetKwProg{Fn}{function}{\string:}{}
\newcommand{\forcond}{$i=0$ \KwTo $n$}
\SetKwFunction{FRecurs}{FnRecursive}
\SetKwInput{KwCost}{Cost}

\usepackage{holtpolt}

%自定义设置
\renewcommand{\proofname}{Proof.}
\renewcommand{\contentsname}{ Content }
\newcommand{\image}[2]{
    \centering
    \includegraphics[width={#1}\textwidth]{#2}
}



\newcommand\keywords[1]{\vskip2ex\par\noindent\normalfont{\textbf{关键词}: #1}}
\newcommand{\ekeywords}[1]{\vskip2ex\par\noindent\normalfont{\bfseries Key Words: }#1}
\newcommand{\miku}{\textcolor{miku}}
\newcommand{\sakura}{\textcolor{sakura}}
\newcommand{\brown}{\textcolor{brow}}
\newcommand{\red}{\textcolor{red}}
\newcommand{\blue}{\textcolor{blue}}
\newcommand{\A}{\mathcal{A}}
\newcommand{\C}{\mathbb{C}}
\newcommand{\al}{\alpha}
\newcommand{\sa}{$\sigma$-algebra}
\newcommand{\Bsa}{Borel $\sigma$-algebra}
\newcommand{\F}{\mathcal{F}}
\newcommand{\N}{\mathcal{N}}
\newcommand{\M}{\mathcal{M}}
\newcommand{\m}{ $\mathcal{M}$ }
\newcommand{\B}{\mathcal{B}}
\newcommand{\myP}{\mathcal{P}}
\renewcommand{\bf}[1]{\textbf{#1}}

\newcommand{\myRom}[1]{\uppercase\expandafter{\romannumeral#1}}
\newcommand{\pl}{$ L^p(X) $}
\newcommand{\twol}{$ L^2(X) $}

\usepackage{booktabs}

\begin{document}
\hfill\vbox{\hbox{NPDE-FDM}\hbox{陈曦,HOME}\hbox{Summer, 2024}}

\begin{center}\Large
    \textbf{微分方程数值解——有限差分法}\\{\normalsize\bf {偏微分方程组}}
\end{center}
\vskip 30pt
\small {参考书目:
\begin{itemize}
    \item Numerical Partial Differential Equations: Finite Difference Methods (J. W. Thomas,1995)
    \item Time Dependent Problems and Difference Methods(B. Gustafsson,1995)
    \item Finite Difference Methods for Ordinary and Partial Differential Equations(Randall J.LeVeque,2007)
    \item 偏微分方程的有限差分方法(张强,2017)
\end{itemize}}
本文主要讨论偏微分方程组的适定性理论,不涉及对数值方法的分析,仅仅考虑问题本身的性质,内容对应TDPDM的第四章。

粗略地来说,所谓适定性是指问题的解同时满足:
\begin{itemize}
    \item 存在性;
    \item 唯一性;
    \item 稳定性:解连续依赖于输入数据。
\end{itemize}
对于偏微分方程组的初值问题
\begin{equation}\label{eq:pde}
    \begin{cases}
        u_t = F(u,x,t), &x\in\Omega, t>0,\\
        u(x,0) = f(x), &x\in\Omega,
    \end{cases}
\end{equation}
其中$ u = u(x,t) $是未知函数,$ \Omega $是定义域,$ F $是给定的函数,$ f $是给定的初值函数。我们希望该问题的解$ u(x,t) $的能量(某范数)可以被初值控制,即
\[
    \| u(\cdot,t) \| \leqslant K \| u(\cdot,0) \| = K \| f \| ,\quad t>0,
\]
此时如果对初值进行扰动,令$ f_\varepsilon = f + \varepsilon g $,则对应的解$ u_\varepsilon $满足
\[
    \| u_\varepsilon(\cdot,t) - u(\cdot,t) \| \leqslant K \| g \|,  
\]
因此数值误差可以被初值误差控制,此时该问题的解连续依赖于初值,所以满足稳定性条件(与差分格式的稳定性是不同概念)。不过一般而言,解无法仅通过常数因子用初值控制,例如考虑带有零阶导数项的方程$ u_t=u $,此时解的范数可能会随时间增长,因此我们需要引入指数增长因子,即
\begin{equation}
    \| u(\cdot,t) \| \leqslant Ke^{\beta t} \| f \|, \quad t>0.
\end{equation}

\section{线性偏微分方程的适定性}
本节考虑线性偏微分方程方程组
\begin{equation}\label{eq:lpde}
    u_t = P( x,t,\dfrac{\partial }{\partial x})u 
\end{equation}
的适定性问题,其中$ P $是线性微分算子
\begin{equation}
    P(x,t,\dfrac{\partial }{\partial x}) = \sum_{|\alpha|\leqslant p} A_\alpha(x,t) \left( \dfrac{\partial }{\partial x^{(1)}}  \right)^{\alpha_1} \cdots \left( \dfrac{\partial }{\partial x^{(n)}}  \right)^{\alpha_n},
\end{equation}
其中$ \alpha = (\alpha_1,\cdots,\alpha_n) $是多重指标,$ |\alpha| = \alpha_1 + \cdots + \alpha_n $,$ A_\alpha(x,t) $是给定的$ m\times m $阶矩阵值函数。要求$ A_\alpha\in C^\infty $,并且考虑在各个空间变量上都具有$ 2\pi $周期边界条件的初值问题,现在给出这一问题的适定性的定义。
\begin{definition}
    考虑偏微分方程\eqref{eq:lpde}的在$ 2\pi $周期边界条件下的初值问题。如果对于任意给定的初值$ f\in C^\infty $,如下两个条件成立:
    \begin{enumerate}
        \item 存在唯一解$ u\in C^\infty(x,t) $,该解是以$ 2\pi $为周期的周期函数;
        \item 存在常数$ \beta,K $,使得对于任意$ t>0 $,有
        \begin{equation}\label{eq:wellposed}
            \| u(\cdot,t) \| \leqslant Ke^{\beta t} \| f \|,
        \end{equation}
        则称该问题是适定的。如果问题不适定,则称其为病态的。
    \end{enumerate}
\end{definition}

根据Parseval定理,$ \| u(x,t) \|_2 = \| \hat{u}(\omega,t) \|_2 $,因此可以使用Fourier变换来研究问题的适定性,将单波解
\begin{equation}
    u(x,t) = \hat{u}(\omega,t)e^{i \omega x}
\end{equation}
(这一单波解对应单波初始条件$ f(x) = e^{i \omega x}\hat{f}(\omega) $,其中$ \hat{f}(\omega) = 1 / (2 \pi) \int_{-\pi}^\pi f(x)e^{-i \omega x}dx $)代入给定的方程\eqref{eq:lpde}可以得到$ \hat{u}(\omega,t) $与$ \hat{u}(\omega,0)=\hat{f} $之间的关系,借助这两者之间增长的关系可以判断问题的适定性。

\begin{example}
    考虑具有周期性边界的二维扩散方程$ u_t = \kappa\Delta u $的初值问题,令$ u(x,t) = \hat{u}(\omega,t)e^{i \omega\cdot x} $,$ \omega,x $都是二维向量,代入方程可以得到
    \[
        \hat{u}_t(\omega,t) = -\kappa |\omega|^2 \hat{u}(\omega,t),
    \]
    这是一个常微分方程,初值为$ \hat{u}(\omega,0) = \hat{f}(\omega) $,于是
    \begin{equation}
        \hat{u}(\omega,t) = \hat{f}(\omega)e^{-\kappa |\omega|^2 t}.
    \end{equation}
    因此
    \[
        \| u(\cdot,t) \|_2 = \| \hat{u}(\omega,t) \|_2 = \| \hat{f}(\omega) \|_2 e^{-\kappa |\omega|^2 t} = \| f \|_2 e^{-\kappa |\omega|^2 t},
    \]
    为了使得\eqref{eq:wellposed}成立,需要$ \beta = -\kappa |\omega|^2 $关于$ \omega\in \mathbb{R} $有上界,因此为了保证适定性需要要求$ \kappa >0 $。
\end{example}

在上面的例子中,尽管当问题具有$ 2\pi $周期性边界时可以只考虑$ \omega\in [-\pi,\pi] $的简谐波,然而对于数值计算而言,舍入误差的存在使得计算时几乎总是会出现高频分量,因此有必要考虑所有的$ \omega\in \mathbb{R} $,要求任意可能的$ \omega $下的$ \beta $都有上界。

\subsection{一维线性常系数标量方程}
考虑如下一维线性常系数标量方程
\begin{equation}\label{eq:1d_parabolic}
    u_t = au_{xx} + bu_x + cu
\end{equation}
的$ (-\pi,\pi) $周期性边界条件下的初值问题,初值条件为$ u(x,0)=f(x) $。根据上一节的定义我们有如下适定性判定定理。
\begin{theorem}
    周期初边值问题\eqref{eq:1d_parabolic}是适定的,当且仅当存在$ \alpha\in \mathbb{R} $使得
    \begin{equation}
        {\rm Re}\ \kappa(\omega) \leqslant \alpha,\quad\forall \omega\in \mathbb{R},
    \end{equation}
    其中$ \kappa(\omega) = -a\omega^2 + ib\omega + c $。
\end{theorem}
\begin{proof}
    考虑单波解$ u(x,t) = \exp(i \omega x)\hat{u}(\omega,t) $,带入方程\eqref{eq:1d_parabolic}可以得到
    \[
        \hat{u}_t(\omega,t) = (-a\omega^2 + ib\omega + c)\hat{u}(\omega,t) = \kappa \hat{u}(\omega,t),
    \]
    相应的初值条件变为$ \hat{u}(\omega,0) = \hat{f}(\omega) $,该ODE问题的解为
    \[
        \hat{u}(\omega,t) = \hat{f}(\omega)e^{\kappa t},
    \]
    根据Parseval定理,$ \| u(x,t) \|_2 = \| \hat{u}(\omega,t) \|_2 $,因此
    \[
        \| u(x,t) \|_2 = e^{{\rm Re}\ \kappa t} \| \hat{f}(\omega) \|_2 = e^{{\rm Re}\ \kappa t} \| u(x,0) \|_2,  
    \]
    根据适定性的定义,需要$ {\rm Re}\ \kappa(\omega) $关于$ \omega\in \mathbb{R} $有上界,即$ {\rm Re}\ \kappa(\omega) \leqslant \alpha $。
\end{proof}

根据如上定理,方程\eqref{eq:1d_parabolic}的适定性依赖于$ \kappa(\omega) $,因为$ a,b,c\in \mathbb{C} $而$ \omega\in \mathbb{R} $,所以
\[
    {\rm Re}\ \kappa(\omega) = -a_r\omega^2 - b_i \omega + c,
\]
于是我们有如下结论:
\begin{enumerate}
    \item 当$ a_r>0 $时,$ \kappa(\omega) $关于$ \omega $有上界,即
    \[
        \kappa = -a_r\omega^2 - b_i \omega + c \leqslant \frac{b_i}{4a_r} +c,\quad \forall \omega\in \mathbb{R}.
    \]
    \item 当$ a_r=0 $时,$ \kappa(\omega) $关于$ \omega $有上界当且仅当$ b_i =  0 $,此时问题适定。
    \item 当$ a_r<0 $时,$ \kappa(\omega) $关于$ \omega $有不存在上界,因此问题总是病态的。
\end{enumerate}
因此总的来说,方程\eqref{eq:1d_parabolic}的适定性主要取决于最高阶项的系数$ a $的实部符号,这一事实对于更高阶的方程也是成立的。

\subsection{一维线性常系数方程组}
考虑一维线性常系数方程组
\begin{equation}\label{eq:1d_hyperbolic}
    u_t = Au_x,
\end{equation}
其中$ A $是$ m $阶方程,$ u = (u_1,\cdots,u_m) $是$ m $维向量值函数,相应的初值问题为$ u(x,0) = f(x) $。关于这一问题有如下适定性判定定理。
\begin{theorem}
    周期性初边值问题\eqref{eq:1d_hyperbolic}是适定的,当且仅当存在一阶系数矩阵$ A $的各特征值都为实数,且具有$ m $个线性无关的特征向量。
\end{theorem}
\begin{proof}
    首先说明必要性。
    设$ \phi $是$ A $的关于特征值$ \lambda $的一个特征向量,即$ A\phi = \lambda \phi $,则直接验证可知方程\eqref{eq:1d_hyperbolic}在初值$ u(x,0) = e^{i \omega x}\phi $下的一个解为
    \begin{equation}
        u(x,t) = e^{i \omega (x + \lambda t)} \phi = e^{i \lambda \omega t}u(x,0),
    \end{equation}
    因此
    \[
        \| u(x,t) \|_2 = e^{-({\rm Im}\ \lambda) \omega t} \| u(x,0) \|_2,
    \]
    为了保证适定性,需要$ {\rm Im}\ \lambda \cdot \omega $关于$ \omega \in \mathbb{R} $有上界,所以$ {\rm Im}\ \lambda = 0 $,因此$ \lambda $只能是实数。
    为了说明$ A $必须有$ m $个线性无关的特征向量,考虑$ A $的Jordan标准型,即
    \[
        S^{-1}AS = 
        \begin{pmatrix} 
            \lambda_1 I + J_1 & & & \\
            & \lambda_2 I + J_2 & & \\ 
            & & \ddots & \\
            & & & \lambda_k I + J_k
        \end{pmatrix}= J ,
    \]
    其中$ \lambda_i $是$ A $的不同特征值,$ J_i $是对应的Jordan块。令$ v = S^{-1}u $,则原方程
    \[
        v_t = S^{-1}u_t = S^{-1}Au_x = S^{-1}AS S^{-1}u_x = Jv_x,
    \]
    考虑单波解$ v(x,t) = e^{i \omega x}\hat{v}(\omega,0) $,则
    \[
        \hat{v}_t(\omega,t) = i \omega J\hat{v}(\omega,t), \quad \hat{v}(\omega,0) = S^{-1}\hat{u}(\omega,0),
    \]
    所以$ \hat{v}(\omega,t) = \exp(i \omega J t)\hat{v}(\omega,0) $,其中
    \[
        \exp(i \omega J t) = \sum_{n=0}^{\infty} \frac{(i \omega t)^n}{n!} J^n,
    \]
    因为$ J $是分块对角阵,因此$ J^n $也是分块对角阵,只需考虑单独的一个Jordan块,于是问题转化为
    \[
        \hat{w} = \exp(i \omega (\lambda_j I+J_j) t)\hat{w}(0),
    \]
    所以
    \[
        \| w \|_2 \leqslant \| \exp(i \omega (\lambda_j I+J_j) t) \|_2 \| w(0) \|_2,
    \]
    为此我们考虑$ \exp(i \omega (\lambda_j I+J_j) t) $的实部矩阵的大小,注意到$ \lambda_j I $与$ J_i $可交换,因此
    \[
        \exp(i \omega (\lambda_j I+J_j) t) = \exp(i \omega \lambda_j t) \exp(i \omega J_j t),
    \]
    因为$ \lambda_j $为实数,因此只需考虑$ \exp(i \omega J_j t) $,注意到$ J_j^{m_j} = 0 $,其中$ m_j $是该Jordan块的大小,展开可得
    \[
        \exp(i \omega J_j t) = \sum_{n=0}^{\infty} \frac{(i \omega t)^n}{n!} J_j^n = \sum_{n=0}^{m_j-1} \frac{(i \omega t)^n}{n!} J_j^n
    \]
    当$ m_j\ne 1 $时,上式关于$ \omega \in \mathbb{R} $无法被控制,因此为了保证适定性,需要$ m_j = 1 $,即$ A $的只有一阶Jordan块,可逆$ S $是$ A $的特征向量矩阵,所以$ A $必须有$ m $个线性无关的特征向量。

    现在说明充分性。当$ A $有$ m $个线性无关的特征向量时,可以将$ A $进行相似对角化$ A = S D S^{-1} $,和之前一样令$ v = S^{-1}u $可以将原方程解耦成$ m $个标量方程:
    \[
        (v_i)_t = \lambda_i(v_i)_x,\quad i=1,\cdots,m,
    \]
    其中$ \lambda_i\in \mathbb{R} $是$ A $的特征值,$ v_i $是$ v $的第$ i $个分量函数。根据上一小节对标量方程的适定性判定,每个标量方程都是适定的,因此转换后的$ v_t = D v_x $是适定的,且
    \[
        \| v(x,t) \|_2 = \| v(x,0) \|_2,
    \]
    现在我们需要将$ v $转换回$ u $,使用范数的相容性可得
    \[
        \| u(x,t) \|_2 = \| Sv(x,t) \|_2 \leqslant \| S \|_2 \| v(x,0) \|_2 = \| S \|_2 \| S^{-1}u(x,0) \|_2 \leqslant \| S \|_2 \| S^{-1} \|_2 \| u(x,0) \|_2,
    \]
    所以原问题是适定的。
\end{proof}

下面我们考虑双曲方程,首先给出一些定义:
\begin{definition}
    考虑形如$ u_t = Au_x $的方程,则
    \begin{enumerate}
        \item 如果$ A $的特征值都是实数,则称该方程是(弱)双曲的(weakly hyperbolic);
        \item 如果$ A = A^* $是Hermite矩阵,则称该方程是对称双曲的(symmetric hyperbolic);
        \item 如果$ A $的特征值都是实数且具有$ m $个线性无关的特征向量,则称该方程是强双曲的(strongly hyperbolic);
        \item 如果$ A $具有$ m $个互不相同的实特征值,则称该方程是严格双曲的(strictly hyperbolic)。
    \end{enumerate}
\end{definition}
显然对于对称双曲、强双曲和严格双曲方程,系数矩阵$ A $都可以进行相似对角化。

在讨论双曲方程的适定性之前,我们先给出一个引理:
\begin{lemma}
    如果$ y\in C^1 $满足
    \begin{equation}
        \frac{dy}{dt} \leqslant \alpha y,\quad \forall t\geqslant 0,
    \end{equation}
    则
    \begin{equation}
        y(t) \leqslant y(0)e^{\alpha t},\quad \forall t\geqslant 0.
    \end{equation}
\end{lemma}
\noindent 证明只需考虑$ v = e^{-\alpha t}y $,它满足$ \frac{dv}{dt} \leqslant 0 $,因此$ v(t) \leqslant v(0) $,于是$ y(t) \leqslant y(0)e^{\alpha t} $。

现在我们考虑更一般的带有零阶项的强双曲方程
\[
    u_t = Au_x + Bu,  
\]
关于这一方程的适定性有如下定理:
\begin{theorem}
    考虑形如$ u_t = Au_x + Bu $的强双曲方程,其中$ A,B $是$ m $阶矩阵,$ A $是强双曲的$ m $阶方阵,$ B $是任意常数矩阵,则该方程的初值问题是适定的。
\end{theorem}
\begin{proof}
    因为$ A $是强双曲的,因此取$ S $为$ A $的线性无关的特征向量组成的矩阵,$ S^{-1}AS = D $是对角矩阵,$ D $的对角元是$ A $的特征值,于是$ v = S^{-1}u $满足
    \[
        v_t = Dv_x + S^{-1}BS v,
    \]
    记$ S^{-1}BS = C $,相应的初值条件为$ v(x,0) = S^{-1}u(x,0) $。考虑单波解$ v(x,t) = e^{i \omega x}\hat{v}(\omega,t) $,则
    \[
        \hat{v}_t(\omega,t) = i \omega D\hat{v}(\omega,t) + C\hat{v}(\omega,t),
    \]
    于是
    \[
        \hat{v}(\omega,t) = \exp((i \omega D+C) t)\hat{v}(\omega,0).
    \]
    注意到
    \[
        \begin{aligned}
            \dfrac{\partial }{\partial t} (\hat{v},\hat{v}) = (\hat{v}_t,\hat{v}) + (\hat{v},\hat{v}_t) 
            &= (i \omega D\hat{v} + C\hat{v},\hat{v}) + (\hat{v},i \omega D\hat{v} + C\hat{v}) \\
            &= (i \omega D \hat{v}, \hat{v}) + (\hat{v},i \omega D\hat{v}) + (C\hat{v},\hat{v}) + (\hat{v},C\hat{v}) \\
        \end{aligned},
    \]
    因为$(\hat{v},i \omega D\hat{v}) = -i \omega (D^* \hat{v}, \hat{v}) = -(i \omega D \hat{v}, \hat{v}) $,所以上式即
    \[
        \dfrac{\partial }{\partial t} (\hat{v},\hat{v}) = (C\hat{v},\hat{v}) + (\hat{v},C\hat{v}) \leqslant 2\| C \|_2 (\hat{v},\hat{v}),
    \]
    进而使用上一个引理两侧关于$ t $积分可得
    \[
        \| \hat{v}(\omega,t) \|_2^2 =(\hat{v}(\omega,t),\hat{v}(\omega,t)) \leqslant (\hat{v}(\omega,0),\hat{v}(\omega,0))e^{2\| C \|_2 t} = \| \hat{v}(\omega,0) \|_2^2 e^{2\| C \|_2 t},
    \]
    因此
    \[
        \| \hat{v}(\omega,t) \|_2^2 = \| \exp((i \omega D+C) t)\hat{v}(\omega,0) \|_2^2 \leqslant \| \hat{v}(\omega,0) \|_2^2 e^{2\| C \|_2 t},
    \]
    这说明
    \[
        \| \exp((i \omega D+C) t) \|_2 \leqslant e^{\| C \|_2 t}.
    \]
    因为方程的通解形如
    \[
        v(x,t) = \sum_{\omega = -\infty}^{\infty} \hat{v}(\omega,t)e^{i \omega x},
    \]
    带入$ \hat{v} $可得
    \[
        v(x,t) = \sum_{\omega = -\infty}^{\infty} e^{i \omega x + (i \omega D+C) t}\hat{v}(\omega,0),
    \]
    使用Parseval定理并使用$ \| \exp((i \omega D+C) t) \|_2 \leqslant e^{\| C \|_2 t} $可得
    \[
        \| v(x,t) \|_2 \leqslant e^{\| C \|_2 t} \| v(x,0) \|_2.
    \]
    最后使用$ u = Sv $可得
    \[
        \| u(x,t) \|_2 \leqslant \| S \|_2 \| S^{-1} \|_2  e^{\| C \|_2 t} \| u(x,0) \|_2,
    \]
    因此原问题是适定的。
\end{proof}

\subsection{一维常系数抛物方程组}
考虑二阶常系数方程组
\begin{equation}\label{eq:parabolics}
    u_t = Au_{xx} + Bu_x + Cu
\end{equation}
的周期性初值问题,初值条件为$ u(x,0) = f(x) $。

\begin{definition}
    如果形如\eqref{eq:parabolics}的方程组中的$ A $的所有特征值的实部都有正的下界,即
    \begin{equation}
        {\rm Re}\ \lambda \geqslant \delta,
    \end{equation}
    其中$ \lambda $是$ A $的特征值,$ \delta $是一个正常数,则称该方程组是抛物的。
\end{definition}

为了简单起见,如果$ A=A^* $是Hermite的,而且$ (Av,v)\geqslant 0 $对任意$ v $都成立,则称$ A\geqslant 0 $。如果$ A,B $都是Hermite矩阵且$ A-B\geqslant 0 $,则记为$ A\geqslant B $。如果\eqref{eq:parabolics}是抛物的,则对$ A $使用Schur分解\footnote{对矩阵阶数进行归纳可证明}可以将它酉上三角化,即
\[
    A = U^*TU,
\]
其中$ U $是酉矩阵,$ T $是上三角矩阵,$ T $的对角元是$ A $的特征值,即
\[
    T = 
    \begin{pmatrix}
        \lambda_1 & t_{12} & \cdots & t_{1m} \\
         & \lambda_2 & \cdots & t_{2m} \\
         & & \ddots & \vdots \\
         & & & \lambda_m
    \end{pmatrix}.
\]
为了控制非对角元的大小,令$ D = \operatorname{diag}\{1,d,\cdots ,d^{m-1}\} $是对角矩阵,考虑
\begin{equation}
    \hat{A} = D^{-1}U^*TUD = 
    \begin{bmatrix} 
        \lambda_1 & d t_{12} & \cdots & d^{m-1} t_{1m} \\
         & \lambda_2 & \cdots & d^{m-2} t_{2m} \\
         & & \ddots & \vdots \\
         & & & \lambda_m 
    \end{bmatrix} = 
    \Lambda + G ,
\end{equation}
其中$ \Lambda $是对角矩阵,$ G $是严格上三角矩阵,当$ d $足够小时,$ \hat{G} $的各个元素都足够接近零,于是
\begin{equation}
    G+G^*+\delta I\geqslant 0,
\end{equation}
而因为方程是抛物的,所以
\begin{equation}
    \Lambda + \Lambda^* = 2{\rm Re}\ \Lambda \geqslant 2\delta I,
\end{equation}
因此
\begin{equation}
    \hat{A} + \hat{A}^* = \Lambda + \Lambda^* + G + G^* \geqslant 2\delta I + G + G^* \geqslant \delta I.
\end{equation}
使用这一事实可以得到如下适定性定理:
\begin{theorem}
    一维常系数抛物方程组\eqref{eq:parabolics}的周期性初值问题是适定的。
\end{theorem}
\begin{proof}
    做变量代换,令$ u = UD v $,其中$ U,D $是之前分析中的矩阵,满足$ \hat{A} = D^{-1}U^*TUD $,于是方程相应地变为
    \[
        v_t = \hat{A}v_{xx} + \hat{B}v_x + \hat{C}v,\quad v(x,0) = D^{-1}U^*f(x),
    \]
    考虑单波解$ v(x,t) = e^{i \omega x}\hat{u}(\omega,t) $,带入方程可以得到
    \[
        \hat{v}_t = (-\omega^2 \hat{A} + i \omega \hat{B} + \hat{C}) \hat{u}, \quad \hat{v}(\omega,0) = D^{-1}U^*\hat{f}(\omega),
    \]
    记$ \hat{P}(i \omega) = - \omega^2 \hat{A} + i \omega \hat{B} + \hat{C} $,则
    \[
        \hat{v}(\omega,t) = e^{\hat{P}(i \omega)t}\hat{f}(\omega).
    \]
    根据之前的分析有
    \[
        \hat{A} + \hat{A}^* \geqslant \delta I,
    \]
    于是
    \begin{equation}
        \begin{aligned}
            \hat{P}(i \omega) + \hat{P}^*(i \omega)  = - \omega^2 (\hat{A} + \hat{A}^*) + i \omega(\hat{B} + \hat{B}^*) + (\hat{C} + \hat{C}^*) 
            &\leqslant  (-\delta \omega^2 + 2 \| B \|_2 \omega + 2 \|\hat{C}\|_2)I \\
            &\leqslant \left( \frac{\|\hat{B}\|_2^2}{\delta} + 2\|\hat{C}\|_2 \right) I,
        \end{aligned}
    \end{equation}
    其中用到了$ \hat{B} + \hat{B}^* \leqslant 2 \| \hat{B} \|_2 I $和$ \hat{C} + \hat{C}^* \leqslant 2 \| \hat{C} \|_2 I $。记上式右侧的上界为$ 2 \alpha I $,于是
    \[
        \hat{P}(i \omega) + \hat{P}^*(i \omega) \leqslant 2 \alpha I.
    \]
    现在与之前一样考虑
    \[
        \begin{aligned}
            \dfrac{\partial }{\partial t} (\hat{v},\hat{v}) = (\hat{v}_t,\hat{v}) + (\hat{v},\hat{v}_t) 
            &= (\hat{P}(i \omega)\hat{v},\hat{v}) + (\hat{v},\hat{P}(i \omega)\hat{v})\\
            &= ((\hat{P}(i \omega)+P^*(i \omega))\hat{v}, \hat{v})\\
            &\leqslant 2\alpha (\hat{v},\hat{v}),
        \end{aligned}
    \]
    再次使用引理可知
    \[
        \| \hat{v}(\omega,t) \|_2^2 \leqslant e^{2\alpha t} \| \hat{v}(\omega,0) \|_2^2 ,  
    \]
    于是
    \[
        \| \hat{v}(\omega,t) \|_2 = \| e^{\hat{P}(i \omega)t} \hat{v}(\omega,0) \|_2 \leqslant e^{\alpha t} \| \hat{v}(\omega,0) \|_2,
    \]
    进而
    \[
        \| e^{\hat{P}(i \omega)t} \|_2 \leqslant e^{\alpha t}.
    \]
    当考虑一般的初值$ f $而非单波初值时,可以将$ f $展开为Fourier级数
    \[
        f(x) = \sum_{\omega = -\infty}^{\infty} \hat{f}(\omega)e^{i \omega x},
    \]
    于是方程的一个形式解为
    \[
        v(x,t) = \sum_{\omega = -\infty}^{\infty} e^{i \omega x + \hat{P}(i \omega)t}\hat{f}(\omega),
    \]
    根据Parseval定理可得
    \[
        \| v(x,t) \|_2^2 = \sum_{\omega = -\infty}^{\infty} \| e^{i \omega x + \hat{P}(i \omega)t} \hat{f}(\omega)\|_2^2 \leqslant \sum_{\omega = -\infty}^{\infty} e^{2\alpha t} \| \hat{f}(\omega) \|_2^2 = e^{2\alpha t} \| f \|_2^2,
    \]
    于是该形式解是连续依赖于初值,现在我们需要将$ v $转换回$ u $,使用范数的相容性可得
    \[
        \| u(x,t) \|_2 \leqslant \| U \|_2 \| D \|_2 e^{\alpha t} \| U^* \|_2 \| D^{-1} \|_2 \| u(x,0) \|_2 = \| D \|_2 \| D^{-1} \|_2 e^{\alpha t} \| f \|_2,
    \]
    因此$ u = D^{-1}U^* v $是连续依赖于初值。最后我们证明$ v $是唯一解,如果$ w $是满足相同初值条件$ v(x,0)=w(x,0) $的另一个解,将$ v-w $进行Fourier展开可得
    \[
        v(x,t) - w(x,t) = \sum_{\omega = -\infty}^{\infty} e^{i \omega x}(\hat{v}(\omega,t) - \hat{w}(\omega,t)),
    \]
    于是
    \[
        \dfrac{\partial }{\partial t} (e^{i \omega x}, v-w) = (e^{i \omega x}, v_t-w_t) = (e^{i \omega x}, \hat{A}(v_{xx}-w_{xx}) + \hat{B}(v_x-w_x) + \hat{C}(v-w)),
    \]
    使用分部积分可知
    \[
        (e^{i \omega x}, v_x-w_x) = i \omega (e^{i \omega x}, v-w),
    \]
    再次微分可知
    \[
        (e^{i \omega x}, v_{xx}-w_{xx}) = -\omega^2 (e^{i \omega x}, v-w),
    \]
    注意到$ (e^{i \omega x}, v-w) = 2\pi (\hat{v}-\hat{w}) $,所以
    \[
        \dfrac{\partial }{\partial t} (\hat{v}-\hat{w}) = \hat{P}(i \omega)(\hat{v}-\hat{w}),\quad \hat{v}(\omega,0) - \hat{w}(\omega,0) = 0,
    \]
    因为上述方程只有零解,所以$ v = w $,因此解是唯一的。综上所述,该问题是适定的。
\end{proof}

\subsection{一般常系数方程组}
考虑一般的常系数方程组的初值问题
\begin{equation}\label{eq:general}
    u_t = P \left( \dfrac{\partial }{\partial t}  \right) u,\quad u(x,0) = f(x),
\end{equation}
其中$ P $是一个常系数多项式,$ \omega = (\omega_1, \cdots , \omega_d) $是实波数向量,$ x = (x_1,\cdots ,x_d)\in \mathbb{R}^d $是空间变量,这种情况下单波初值为
\[
    f(x) = e^{i \omega \cdot x} \hat{f}(\omega),
\]
相应地单波解为
\begin{equation}
    u(x,t) = e^{i \omega \cdot x} \hat{u}(\omega,t),
\end{equation}
与之前一样将该单波解带入方程可以得到
\begin{equation}
    \hat{u}_t(\omega,t) = P(i \omega) \hat{u}(\omega,t), \quad \hat{u}(\omega,0) = \hat{f}(\omega),
\end{equation}
其中的$ P(i \omega) $是$ m $阶矩阵,其中的元素是$ i \omega_j $的多项式。对于这一问题我们有如下结论
\begin{theorem}
    一般常系数方程组\eqref{eq:general}的周期性初值问题是适定的当且仅当存在常数$ K,\alpha $使得
    \begin{equation}
        \| e^{\hat{P}(i \omega)t} \|_2 \leqslant K e^{\alpha t},\quad \forall \omega\in \mathbb{R}^d.
    \end{equation}
\end{theorem}
证明可以视作之前各特例的推广。在实际应用中我们更常使用如下的必要性条件:
\begin{theorem}{\normalfont\bf{The Petrovskii condition}}
    方程组\eqref{eq:general}的周期性初值问题适定的一个必要条件是$ \hat{P}(i \omega) $的任意特征值$ \lambda $的实部都关于$ \omega\in \mathbb{R}^d $有上界,即
    \begin{equation}
        {\rm Re}\ \lambda \leqslant \alpha,\quad \forall \omega\in \mathbb{R}^d.
    \end{equation}
\end{theorem}
在如上定理的基础上我们有下面的充分性定理:
\begin{theorem}
    如果Petrovskii条件满足,且存在常数$ K $和线性变换$ S(\omega) $满足
    \begin{equation}
        \| S(\omega) \|_2 \| S^{-1}(\omega) \|_2 \leqslant K,\quad \forall \omega\in \mathbb{R}^d
    \end{equation}
    使得$ S^{-1}(\omega)\hat{P}(i \omega)S(\omega) $是对角矩阵,则方程组\eqref{eq:general}的周期性初值问题是适定的。
\end{theorem}
作为一个推论有
\begin{corollary}
    如果存在常数$ \alpha $使得
    \begin{equation}
        \hat{P}(i \omega) + \hat{P}^*(i \omega) \leqslant 2\alpha I,\quad \forall \omega\in \mathbb{R}^d,
    \end{equation}
    则方程组\eqref{eq:general}的周期性初值问题是适定的。
\end{corollary}

\end{document}