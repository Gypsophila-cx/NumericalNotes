\documentclass[a4paper,10pt]{ctexart}
%引用设置使用Bibtex
\usepackage{gbt7714}
\bibliographystyle{gbt7714-numerical}
%页面设置
\usepackage{geometry}
%字体设置
\usepackage{fontspec}
%\setmainfont{Times New Roman}
%定理环境
\usepackage{amsmath}
\numberwithin{equation}{section}
\usepackage{amsthm}
\newtheorem*{definition}{Definition}
\newtheorem{theorem}{Theorem}
\newtheorem{lemma}{Lemma}
\newtheorem*{corollary}{Corollary}
\newtheorem*{proposition}{Proposition}
\newtheorem*{example}{Example}
%数学环境字体
\usepackage{bm}
\usepackage[all]{xy}
%加载 TikZ 用于绘制交换图
\usepackage{tikz-cd}
\usepackage{tikz}
\usepackage{pgfplots}
\newcommand{\tikzdef}{\pgfmathsetmacro} % 在tikzpicture内的foreach循环中定义实数临时变量
%颜色
\usepackage{color,xcolor}

\definecolor{miku}{RGB}{57,197,187}
\definecolor{sakura}{RGB}{255,192,203}
\definecolor{rose}{RGB}{255,228,225}
\definecolor{brown}{RGB}{210,105,30}
\definecolor{lbrown}{RGB}{239,235,224}
\definecolor{bule}{RGB}{0,47,167}
\definecolor{lyellow}{RGB}{250,250,210}
\definecolor{lpurple}{RGB}{255,240,245}
\definecolor{lbule}{RGB}{135,206,250}
\definecolor{gbule}{RGB}{64,224,208}
\definecolor{green}{RGB}{138,200,207}
\definecolor{lgreen}{RGB}{225,255,255}
\definecolor{lorange}{RGB}{248,172,140}
\definecolor{salmon}{RGB}{250,128,114}
\definecolor{burgundy}{rgb}{0.5, 0.0, 0.13}
%链接设置
\usepackage[colorlinks=true,pdfstartview=FitH,linkcolor=blue,anchorcolor=violet, citecolor=magenta]{hyperref} 
%封面
\usepackage{pdfpages}
\usepackage{mathrsfs}
\usepackage{amssymb}
\usepackage{graphicx}
\usepackage{lipsum}
%彩色框
\usepackage{framed}
\usepackage{tcolorbox}
\tcbuselibrary{breakable}
\tcbuselibrary{theorems}
\tcbuselibrary{skins}
\usepackage{colortbl}
\usepackage{float}
\usepackage[export]{adjustbox}
\newtcolorbox[auto counter,number within=section]{notebox}[2][]{%
colback=miku!2!white,
colframe=miku,
coltitle=white,
fonttitle=\bfseries,
rightrule=2pt,
leftrule=2pt,
bottomrule=2pt,
colbacktitle=miku,
theorem style=standard,
breakable,
arc=2pt,
drop fuzzy shadow=black!20!white,
title=Note~\thetcbcounter: #2,#1}
\newtcolorbox[auto counter,number within=section]{markbox}[2][]{%
colback=miku!2!white,
colframe=miku,
coltitle=white,
fonttitle=\bfseries,
rightrule=0pt,
leftrule=0pt,
bottomrule=2pt,
colbacktitle=miku,
theorem style=standard,
breakable,
arc=0pt,
drop fuzzy shadow=black!20!white,
title=Remark~\thetcbcounter: #2,#1}
\newtcolorbox[no counter]{theorems}[2][]{%
width=12cm,
center,
sidebyside,
sidebyside adapt=left,
sidebyside gap=6mm,
sidebyside align=center seam,
colback=burgundy!2!white,
colframe=burgundy,
coltitle=white,
fonttitle=\bfseries,
rightrule=1pt,
leftrule=1pt,
bottomrule=2pt,
colbacktitle=burgundy,
theorem style=standard,
enhanced,
drop fuzzy shadow southeast=black!30!white,
breakable,
arc=0pt,
title=Theorem. #2,#1}
\newtcolorbox[no counter]{definitions}[2][]{%
width=12cm,
center,
colback=lyellow!2!white,
colframe=yellow!3!lyellow,
coltitle=bule,
fonttitle=\bfseries,
rightrule=0pt,
leftrule=1pt,
bottomrule=2pt,
colbacktitle=lyellow,
theorem style=standard,
breakable,
arc=5pt,
enhanced,
drop fuzzy shadow southeast=black!20!white,
title=Definition. #2,#1}
\newtcolorbox[auto counter,number within=section]{corollarys}[2][]{%
colback=lyellow!2!white,
colframe=lyellow,
coltitle=bule,
fonttitle=\bfseries,
rightrule=0pt,
leftrule=1pt,
bottomrule=2pt,
colbacktitle=lyellow,
theorem style=standard,
breakable,
arc=0pt,
enhanced,
drop fuzzy shadow southeast=black!20!white,
title=Corollary~\thetcbcounter: #2,#1}
\newtcolorbox[auto counter,number within=section]{lemmas}[2][]{%
width=12cm,
center,
colback=lyellow!2!white,
colframe=lorange!30!sakura,
coltitle=bule,
fonttitle=\bfseries,
rightrule=0pt,
leftrule=1pt,
bottomrule=2pt,
colbacktitle=lorange!30!sakura,
theorem style=standard,
breakable,
arc=5pt,
enhanced,
drop fuzzy shadow southeast=black!20!white,
title=Lemma. #2,#1}
\newtcolorbox[auto counter,number within=section]{propositions}[2][]{%
width=12cm,
center,
colback=salmon!5,
colframe=salmon!90!black,
coltitle=white,
fonttitle=\bfseries,
rightrule=1pt,
leftrule=1pt,
bottomrule=2pt,
colbacktitle=salmon!90!black,
theorem style=standard,
breakable,
arc=5pt,
enhanced,
drop fuzzy shadow southeast=black!20!white,
title=Proposition. #2,#1}
\newtcolorbox[no counter]{egbox}[2][]{%
width=12cm,
center,
colback=black!5!white,
colframe=black!20!white,
coltitle=black,
fonttitle=\bfseries,
rightrule=1pt,
leftrule=1pt,
bottomrule=2pt,
colbacktitle=black!20!white,
theorem style=standard,
breakable,
arc=0pt,
enhanced,
drop fuzzy shadow southeast=black!20!white,
title=Example. #2,#1}

%\begin{figure}[H]
%\centering
%\includegraphics[center]{pic.png}
%\end{figure}
\geometry{left=3cm,right=3cm,top=2cm,bottom=2cm}
\tcbuselibrary{most}

\usepackage[linesnumbered,ruled,vlined]{algorithm2e}
\usepackage{algorithmic}

\SetKwProg{Fn}{function}{\string:}{}
\newcommand{\forcond}{$i=0$ \KwTo $n$}
\SetKwFunction{FRecurs}{FnRecursive}
\SetKwInput{KwCost}{Cost}

\usepackage{holtpolt}

%自定义设置
\renewcommand{\proofname}{Proof.}
\renewcommand{\contentsname}{ Content }
\newcommand{\image}[2]{
    \centering
    \includegraphics[width={#1}\textwidth]{#2}
}



\newcommand\keywords[1]{\vskip2ex\par\noindent\normalfont{\textbf{关键词}: #1}}
\newcommand{\ekeywords}[1]{\vskip2ex\par\noindent\normalfont{\bfseries Key Words: }#1}
\newcommand{\miku}{\textcolor{miku}}
\newcommand{\sakura}{\textcolor{sakura}}
\newcommand{\brown}{\textcolor{brow}}
\newcommand{\red}{\textcolor{red}}
\newcommand{\blue}{\textcolor{blue}}
\newcommand{\A}{\mathcal{A}}
\newcommand{\C}{\mathbb{C}}
\newcommand{\al}{\alpha}
\newcommand{\sa}{$\sigma$-algebra}
\newcommand{\Bsa}{Borel $\sigma$-algebra}
\newcommand{\F}{\mathcal{F}}
\newcommand{\N}{\mathcal{N}}
\newcommand{\M}{\mathcal{M}}
\newcommand{\m}{ $\mathcal{M}$ }
\newcommand{\B}{\mathcal{B}}
\newcommand{\myP}{\mathcal{P}}
\renewcommand{\bf}[1]{\textbf{#1}}

\newcommand{\myRom}[1]{\uppercase\expandafter{\romannumeral#1}}
\newcommand{\pl}{$ L^p(X) $}
\newcommand{\twol}{$ L^2(X) $}

\usepackage{booktabs}

\begin{document}
\hfill\vbox{\hbox{NPDE-FDM}\hbox{陈曦,HOME}\hbox{Summer, 2024}}

\begin{center}\Large
    \textbf{微分方程数值解——有限差分法}\\{\normalsize\bf {偏微分方程组}}
\end{center}
\vskip 30pt
\small {参考书目:
\begin{itemize}
    \item Numerical Partial Differential Equations: Finite Difference Methods (J. W. Thomas,1995)
    \item Time Dependent Problems and Difference Methods(B. Gustafsson,1995)
    \item Finite Difference Methods for Ordinary and Partial Differential Equations(Randall J.LeVeque,2007)
    \item 偏微分方程的有限差分方法(张强,2017)
\end{itemize}}
本文主要讨论偏微分方程组的适定性理论,不涉及对数值方法的分析,仅仅考虑问题本身的性质,内容对应TDPDM的第四章。

粗略地来说,所谓适定性是指问题的解同时满足:
\begin{itemize}
    \item 存在性;
    \item 唯一性;
    \item 稳定性:解连续依赖于输入数据。
\end{itemize}
对于偏微分方程组的初值问题
\begin{equation}\label{eq:pde}
    \begin{cases}
        u_t = F(u,x,t), &x\in\Omega, t>0,\\
        u(x,0) = f(x), &x\in\Omega,
    \end{cases}
\end{equation}
其中$ u = u(x,t) $是未知函数,$ \Omega $是定义域,$ F $是给定的函数,$ f $是给定的初值函数。我们希望该问题的解$ u(x,t) $的能量(某范数)可以被初值控制,即
\[
    \| u(\cdot,t) \| \leqslant K \| u(\cdot,0) \| = K \| f \| ,\quad t>0,
\]
此时如果对初值进行扰动,令$ f_\varepsilon = f + \varepsilon g $,则对应的解$ u_\varepsilon $满足
\[
    \| u_\varepsilon(\cdot,t) - u(\cdot,t) \| \leqslant K \| g \|,  
\]
因此数值误差可以被初值误差控制,此时该问题的解连续依赖于初值,所以满足稳定性条件(与差分格式的稳定性是不同概念)。不过一般而言,解无法仅通过常数因子用初值控制,例如考虑带有零阶导数项的方程$ u_t=u $,此时解的范数可能会随时间增长,因此我们需要引入指数增长因子,即
\begin{equation}
    \| u(\cdot,t) \| \leqslant Ke^{\beta t} \| f \|, \quad t>0.
\end{equation}

\section{线性偏微分方程的适定性}
本节考虑线性偏微分方程方程组
\begin{equation}\label{eq:lpde}
    u_t = P( x,t,\dfrac{\partial }{\partial x})u 
\end{equation}
的适定性问题,其中$ P $是线性微分算子
\begin{equation}
    P(x,t,\dfrac{\partial }{\partial x}) = \sum_{|\alpha|\leqslant p} A_\alpha(x,t) \left( \dfrac{\partial }{\partial x^{(1)}}  \right)^{\alpha_1} \cdots \left( \dfrac{\partial }{\partial x^{(n)}}  \right)^{\alpha_n},
\end{equation}
其中$ \alpha = (\alpha_1,\cdots,\alpha_n) $是多重指标,$ |\alpha| = \alpha_1 + \cdots + \alpha_n $,$ A_\alpha(x,t) $是给定的$ m\times m $阶矩阵值函数。要求$ A_\alpha\in C^\infty $,并且考虑在各个空间变量上都具有$ 2\pi $周期边界条件的初值问题,现在给出这一问题的适定性的定义。
\begin{definition}
    考虑偏微分方程\eqref{eq:lpde}的在$ 2\pi $周期边界条件下的初值问题。如果对于任意给定的初值$ f\in C^\infty $,如下两个条件成立:
    \begin{enumerate}
        \item 存在唯一解$ u\in C^\infty(x,t) $,该解是以$ 2\pi $为周期的周期函数;
        \item 存在常数$ \beta,K $,使得对于任意$ t>0 $,有
        \begin{equation}\label{eq:wellposed}
            \| u(\cdot,t) \| \leqslant Ke^{\beta t} \| f \|,
        \end{equation}
        则称该问题是适定的。如果问题不适定,则称其为病态的。
    \end{enumerate}
\end{definition}

根据Parseval定理,$ \| u(x,t) \|_2 = \| \hat{u}(\omega,t) \|_2 $,因此可以使用Fourier变换来研究问题的适定性,将简谐波解
\begin{equation}
    u(x,t) = \hat{u}(\omega,t)e^{i \omega x}
\end{equation}
代入给定的方程\eqref{eq:lpde}可以得到$ \hat{u}(\omega,t) $与$ \hat{u}(\omega,0)=\hat{f} $之间的关系,借助这两者之间增长的关系可以判断问题的适定性。

\begin{example}
    考虑具有周期性边界的二维扩散方程$ u_t = \kappa\Delta u $的初值问题,令$ u(x,t) = \hat{u}(\omega,t)e^{i \omega\cdot x} $,$ \omega,x $都是二维向量,代入方程可以得到
    \[
        \hat{u}_t(\omega,t) = -\kappa |\omega|^2 \hat{u}(\omega,t),
    \]
    这是一个常微分方程,初值为$ \hat{u}(\omega,0) = \hat{f}(\omega) $,于是
    \begin{equation}
        \hat{u}(\omega,t) = \hat{f}(\omega)e^{-\kappa |\omega|^2 t}.
    \end{equation}
    因此
    \[
        \| u(\cdot,t) \|_2 = \| \hat{u}(\omega,t) \|_2 = \| \hat{f}(\omega) \|_2 e^{-\kappa |\omega|^2 t} = \| f \|_2 e^{-\kappa |\omega|^2 t},
    \]
    为了使得\eqref{eq:wellposed}成立,需要$ \beta = -\kappa |\omega|^2 $关于$ \omega\in \mathbb{R} $有上界,因此为了保证适定性需要要求$ \kappa >0 $。
\end{example}

在上面的例子中,尽管当问题具有$ 2\pi $周期性边界时可以只考虑$ \omega\in [-\pi,\pi] $的简谐波,然而对于数值计算而言,舍入误差的存在使得计算时几乎总是会出现高频分量,因此有必要考虑所有的$ \omega\in \mathbb{R} $,要求任意可能的$ \omega $下的$ \beta $都有上界。

\end{document}