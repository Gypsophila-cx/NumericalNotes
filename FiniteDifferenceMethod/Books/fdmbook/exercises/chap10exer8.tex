
\exercise[(computing with Lax-Wendroff and upwind)]{10.8}

The m-file \verb+advection_LW_pbc.m+ implements the Lax-Wendroff method for
the advection equation on $0\leq x\leq 1$ with periodic boundary conditions.

\begin{enumerate}
\item Observe how this behaves with $m+1 = 50,~100,~200$ grid points.
Change the final time to {\tt tfinal = 0.1} and use
the m-files \verb+error_table.m+ and \verb+error_loglog.m+ to verify second
order accuracy.  

\item Modify the m-file to create a version \verb+advection_up_pbc.m+
implementing the upwind method and verify that this is first order accurate.

\item Keep $m$ fixed and observe what happens with \verb+advection_up_pbc.m+
if the time step $k$ is reduced, e.g. try $k = 0.4h$, $k= 0.2h$, $k=0.1h$.
When a convergent method is applied to
an ODE we expect better accuracy as the time step is reduced and we can
view the upwind method as an ODE solver applied to an MOL system.  However,
you should observe decreased accuracy as $k\goto 0$ with $h$ fixed.  Explain
this apparent paradox.  
{\bf Hint:} What ODE system are we solving more accuracy?  You might also
consider the modified equation (10.44).

\end{enumerate}

