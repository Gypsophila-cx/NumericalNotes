
\exercise[(Pad\'e approximations)]{5.19}

A rational function $R(z) = P(z)/Q(z)$ with 
degree $m$ in the numerator and degree $n$
in the denominator is called the $(m,n)$ {\em Pad\'e approximation}
to a function $f(z)$ if
\[
R(z) - f(z) = \bigo(z^q)
\]
with $q$ as large as possible.  The Pad\'e approximation can be uniquely
determined by expanding $f(z)$ in a Taylor series about $z=0$ and then
considering the series
\[
P(z) - Q(z)f(z),
\]
collecting powers of $z$, and choosing the coefficients of $P$ and $Q$ to
make as many terms vanish as possible.  Trying to require that they all
vanish will give a system of infintely many linear equations for the
coefficients.  Typically these can not all be satisfied simultaneously, 
while requiring the maximal number to hold will give a nonsingular linear
system.  (Note that the $(m,0)$ Pad\'e approximation is simply the first
$m+1$ terms of the Taylor series.)

\vskip 5pt
For this exercise, consider the exponential function $f(z) = e^z$. 
\begin{enumerate}
\item Determine the $(1,1)$ Pad\'e approximation of the form 
\[
R(z) = \frac{1 + a_1 z}{1 + b_1 z}.
\]
Note that this rational function arises from the trapezoidal method applied
to $u' = \lambda u$ (see Exercise 5.18).
\item Determine the $(1,2)$ Pad\'e approximation of the form 
\[
R(z) = \frac{1 + a_1 z}{1 + b_1 z + b_2 z^2}.
\]
\item Determine the $(2,2)$ Pad\'e approximation of the form
\[
R(z) = \frac{1 + a_1 z + a_2 z^2}{1 + b_1 z + b_2 z^2}.
\]
\end{enumerate}
You can check your answers at {\tt
http://mathworld.wolfram.com/PadeApproximant.html}, for example.


