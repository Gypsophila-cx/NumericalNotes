
\exercise[($R(z)$ for Runge-Kutta methods)]{5.18}

Any $r$-stage Runge-Kutta method applied to $u'=\lambda u$ will give an
expression of the form
\[
U^{n+1} = R(z)U^n
\]
where $z=\lambda k$ and $R(z)$ is a rational function, a ratio of
polynomials in $z$ each having degree at most $r$.  For an explicit method
$R(z)$ will simply be a polynomial of degree $r$ and for an implicit method
it will be a more general rational function.

Since $u(t_{n+1}) = e^z u(t_n)$ for this problem, we expect that a $p$th
order accurate method will give a function $R(z)$ satisfying
\eqlex{a}
\qquad  R(z) = e^z + \bigo(z^{p+1}) \quad\text{as}~z \goto 0,
\end{equation}
as discussed in the Remark on page 129.  The rational function $R(z)$ also
plays a role in stability analysis as discussed in Section 7.6.2.

One can determine the value of $p$ in \eqnex{a}. 
by expanding $e^z$ in a Taylor
series about $z=0$, writing the $\bigo(z^{p+1})$ term as
\[
Cz^{p+1} + \bigo(z^{p+2}),
\]
multiplying through by the denominator of $R(z)$, and then collecting terms.
For example, for the trapezoidal method of Exercise 5.17,
\[
\frac{1+z/2}{1-z/2} = \left(1+z+\frac 1 2 z^2 + \frac 1 6 z^3 + \cdots\right)
+Cz^{p+1} + \bigo(z^{p+2})
\]
gives
\begin{equation*}
\begin{split}
1+\half z &= \left(1-\half z\right)\left( 1+z+\half z^2 + \frac 1 6 z^3 +
\cdots\right) + Cz^{p+1} + \bigo(z^{p+2})\\
&= 1 + \half z - \frac{1}{12} z^3 + \cdots + Cz^{p+1}  + \bigo(z^{p+2})
\end{split}
\end{equation*}
and so
\[
Cz^{p+1} = \frac{1}{12} z^3 + \cdots,
\]
from which we conclude that $p=2$.

\begin{enumerate}
\item Let 
\[
R(z) = \frac{1 + \frac 1 3 z}{1 - \frac 2 3 z + \frac 1 6 z^2}.
\]
Determine $p$ for this rational function as an approximation to $e^z$.

\item Determine $R(z)$ and $p$ for the backward Euler method.

\item Determine  $R(z)$ and $p$ for the TR-BDF2 method (5.36).
\end{enumerate}



