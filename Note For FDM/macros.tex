%引用设置使用Bibtex
\usepackage{gbt7714}
\bibliographystyle{gbt7714-numerical}
%页面设置
\usepackage{geometry}
%字体设置
\usepackage{fontspec}
%\setmainfont{Times New Roman}
%定理环境
\usepackage{amsmath}
%\numberwithin{equation}{section}
\usepackage{amsthm}
\newtheorem*{definition}{Definition}
\newtheorem*{theorem}{Theorem}
\newtheorem*{corollary}{Corollary}
\newtheorem*{proposition}{Proposition}
\newtheorem*{example}{Example}
%数学环境字体
\usepackage{bm}
\usepackage[all]{xy}
%加载 TikZ 用于绘制交换图
\usepackage{tikz-cd}
%颜色
\usepackage{color,xcolor}

\definecolor{miku}{RGB}{57,197,187}
\definecolor{sakura}{RGB}{255,192,203}
\definecolor{rose}{RGB}{255,228,225}
\definecolor{brown}{RGB}{210,105,30}
\definecolor{lbrown}{RGB}{239,235,224}
\definecolor{bule}{RGB}{0,47,167}
\definecolor{lyellow}{RGB}{250,250,210}
\definecolor{lpurple}{RGB}{255,240,245}
\definecolor{lbule}{RGB}{135,206,250}
\definecolor{gbule}{RGB}{64,224,208}
\definecolor{green}{RGB}{138,200,207}
\definecolor{lgreen}{RGB}{225,255,255}
\definecolor{lorange}{RGB}{248,172,140}
\definecolor{salmon}{RGB}{250,128,114}
\definecolor{burgundy}{rgb}{0.5, 0.0, 0.13}
%链接设置
\usepackage[colorlinks=true,pdfstartview=FitH,linkcolor=blue,anchorcolor=violet, citecolor=magenta]{hyperref} 
%封面
\usepackage{pdfpages}
\usepackage{mathrsfs}
\usepackage{amssymb}
\usepackage{graphicx}
\usepackage{lipsum}
%彩色框
\usepackage{framed}
\usepackage{tcolorbox}
\tcbuselibrary{breakable}
\tcbuselibrary{theorems}
\tcbuselibrary{skins}
\usepackage{colortbl}
\usepackage{float}
\usepackage[export]{adjustbox}
\newtcolorbox[auto counter,number within=section]{notebox}[2][]{%
colback=miku!2!white,
colframe=miku,
coltitle=white,
fonttitle=\bfseries,
rightrule=2pt,
leftrule=2pt,
bottomrule=2pt,
colbacktitle=miku,
theorem style=standard,
breakable,
arc=2pt,
drop fuzzy shadow=black!20!white,
title=Note~\thetcbcounter: #2,#1}
\newtcolorbox[auto counter,number within=section]{markbox}[2][]{%
colback=miku!2!white,
colframe=miku,
coltitle=white,
fonttitle=\bfseries,
rightrule=0pt,
leftrule=0pt,
bottomrule=2pt,
colbacktitle=miku,
theorem style=standard,
breakable,
arc=0pt,
drop fuzzy shadow=black!20!white,
title=Remark~\thetcbcounter: #2,#1}
\newtcolorbox[no counter]{theorems}[2][]{%
width=12cm,
center,
sidebyside,
sidebyside adapt=left,
sidebyside gap=6mm,
sidebyside align=center seam,
colback=burgundy!2!white,
colframe=burgundy,
coltitle=white,
fonttitle=\bfseries,
rightrule=1pt,
leftrule=1pt,
bottomrule=2pt,
colbacktitle=burgundy,
theorem style=standard,
enhanced,
drop fuzzy shadow southeast=black!30!white,
breakable,
arc=0pt,
title=Theorem. #2,#1}
\newtcolorbox[no counter]{definitions}[2][]{%
width=12cm,
center,
colback=lyellow!2!white,
colframe=yellow!3!lyellow,
coltitle=bule,
fonttitle=\bfseries,
rightrule=0pt,
leftrule=1pt,
bottomrule=2pt,
colbacktitle=lyellow,
theorem style=standard,
breakable,
arc=5pt,
enhanced,
drop fuzzy shadow southeast=black!20!white,
title=Definition. #2,#1}
\newtcolorbox[auto counter,number within=section]{corollarys}[2][]{%
colback=lyellow!2!white,
colframe=lyellow,
coltitle=bule,
fonttitle=\bfseries,
rightrule=0pt,
leftrule=1pt,
bottomrule=2pt,
colbacktitle=lyellow,
theorem style=standard,
breakable,
arc=0pt,
enhanced,
drop fuzzy shadow southeast=black!20!white,
title=Corollary~\thetcbcounter: #2,#1}
\newtcolorbox[auto counter,number within=section]{lemmas}[2][]{%
width=12cm,
center,
colback=lyellow!2!white,
colframe=lorange!30!sakura,
coltitle=bule,
fonttitle=\bfseries,
rightrule=0pt,
leftrule=1pt,
bottomrule=2pt,
colbacktitle=lorange!30!sakura,
theorem style=standard,
breakable,
arc=5pt,
enhanced,
drop fuzzy shadow southeast=black!20!white,
title=Lemma. #2,#1}
\newtcolorbox[auto counter,number within=section]{propositions}[2][]{%
width=12cm,
center,
colback=salmon!5,
colframe=salmon!90!black,
coltitle=white,
fonttitle=\bfseries,
rightrule=1pt,
leftrule=1pt,
bottomrule=2pt,
colbacktitle=salmon!90!black,
theorem style=standard,
breakable,
arc=5pt,
enhanced,
drop fuzzy shadow southeast=black!20!white,
title=Proposition. #2,#1}
\newtcolorbox[no counter]{egbox}[2][]{%
width=12cm,
center,
colback=black!5!white,
colframe=black!20!white,
coltitle=black,
fonttitle=\bfseries,
rightrule=1pt,
leftrule=1pt,
bottomrule=2pt,
colbacktitle=black!20!white,
theorem style=standard,
breakable,
arc=0pt,
enhanced,
drop fuzzy shadow southeast=black!20!white,
title=Example. #2,#1}

%\begin{figure}[H]
%\centering
%\includegraphics[center]{pic.png}
%\end{figure}
\geometry{left=3cm,right=3cm,top=2cm,bottom=2cm}
\tcbuselibrary{most}

%自定义设置
\renewcommand{\proofname}{Proof.}
\renewcommand{\contentsname}{ Content }
\newcommand{\image}[2]{
    \centering
    \includegraphics[width={#1}\textwidth]{#2}
}



\newcommand\keywords[1]{\vskip2ex\par\noindent\normalfont{\textbf{关键词}: #1}}
\newcommand{\ekeywords}[1]{\vskip2ex\par\noindent\normalfont{\bfseries Key Words: }#1}
\newcommand{\miku}{\textcolor{miku}}
\newcommand{\sakura}{\textcolor{sakura}}
\newcommand{\brown}{\textcolor{brow}}
\newcommand{\red}{\textcolor{red}}
\newcommand{\blue}{\textcolor{blue}}
\newcommand{\A}{\mathcal{A}}
\newcommand{\C}{\mathbb{C}}
\newcommand{\al}{\alpha}
\newcommand{\sa}{$\sigma$-algebra}
\newcommand{\Bsa}{Borel $\sigma$-algebra}
\newcommand{\F}{\mathcal{F}}
\newcommand{\N}{\mathcal{N}}
\newcommand{\M}{\mathcal{M}}
\newcommand{\m}{ $\mathcal{M}$ }
\newcommand{\B}{\mathcal{B}}
\newcommand{\myP}{\mathcal{P}}
\renewcommand{\bf}[1]{\textbf{#1}}

\newcommand{\myRom}[1]{\uppercase\expandafter{\romannumeral#1}}
\newcommand{\pl}{$ L^p(X) $}
\newcommand{\twol}{$ L^2(X) $}