\documentclass[a4paper,10pt]{ctexart}
%引用设置使用Bibtex
\usepackage{gbt7714}
\bibliographystyle{gbt7714-numerical}
%页面设置
\usepackage{geometry}
%字体设置
\usepackage{fontspec}
%\setmainfont{Times New Roman}
%定理环境
\usepackage{amsmath}
\numberwithin{equation}{section}
\usepackage{amsthm}
\newtheorem*{definition}{Definition}
\newtheorem{theorem}{Theorem}
\newtheorem{lemma}{Lemma}
\newtheorem*{corollary}{Corollary}
\newtheorem*{proposition}{Proposition}
\newtheorem*{example}{Example}
%数学环境字体
\usepackage{bm}
\usepackage[all]{xy}
%加载 TikZ 用于绘制交换图
\usepackage{tikz-cd}
\usepackage{tikz}
\usepackage{pgfplots}
\newcommand{\tikzdef}{\pgfmathsetmacro} % 在tikzpicture内的foreach循环中定义实数临时变量
%颜色
\usepackage{color,xcolor}

\definecolor{miku}{RGB}{57,197,187}
\definecolor{sakura}{RGB}{255,192,203}
\definecolor{rose}{RGB}{255,228,225}
\definecolor{brown}{RGB}{210,105,30}
\definecolor{lbrown}{RGB}{239,235,224}
\definecolor{bule}{RGB}{0,47,167}
\definecolor{lyellow}{RGB}{250,250,210}
\definecolor{lpurple}{RGB}{255,240,245}
\definecolor{lbule}{RGB}{135,206,250}
\definecolor{gbule}{RGB}{64,224,208}
\definecolor{green}{RGB}{138,200,207}
\definecolor{lgreen}{RGB}{225,255,255}
\definecolor{lorange}{RGB}{248,172,140}
\definecolor{salmon}{RGB}{250,128,114}
\definecolor{burgundy}{rgb}{0.5, 0.0, 0.13}
%链接设置
\usepackage[colorlinks=true,pdfstartview=FitH,linkcolor=blue,anchorcolor=violet, citecolor=magenta]{hyperref} 
%封面
\usepackage{pdfpages}
\usepackage{mathrsfs}
\usepackage{amssymb}
\usepackage{graphicx}
\usepackage{lipsum}
%彩色框
\usepackage{framed}
\usepackage{tcolorbox}
\tcbuselibrary{breakable}
\tcbuselibrary{theorems}
\tcbuselibrary{skins}
\usepackage{colortbl}
\usepackage{float}
\usepackage[export]{adjustbox}
\newtcolorbox[auto counter,number within=section]{notebox}[2][]{%
colback=miku!2!white,
colframe=miku,
coltitle=white,
fonttitle=\bfseries,
rightrule=2pt,
leftrule=2pt,
bottomrule=2pt,
colbacktitle=miku,
theorem style=standard,
breakable,
arc=2pt,
drop fuzzy shadow=black!20!white,
title=Note~\thetcbcounter: #2,#1}
\newtcolorbox[auto counter,number within=section]{markbox}[2][]{%
colback=miku!2!white,
colframe=miku,
coltitle=white,
fonttitle=\bfseries,
rightrule=0pt,
leftrule=0pt,
bottomrule=2pt,
colbacktitle=miku,
theorem style=standard,
breakable,
arc=0pt,
drop fuzzy shadow=black!20!white,
title=Remark~\thetcbcounter: #2,#1}
\newtcolorbox[no counter]{theorems}[2][]{%
width=12cm,
center,
sidebyside,
sidebyside adapt=left,
sidebyside gap=6mm,
sidebyside align=center seam,
colback=burgundy!2!white,
colframe=burgundy,
coltitle=white,
fonttitle=\bfseries,
rightrule=1pt,
leftrule=1pt,
bottomrule=2pt,
colbacktitle=burgundy,
theorem style=standard,
enhanced,
drop fuzzy shadow southeast=black!30!white,
breakable,
arc=0pt,
title=Theorem. #2,#1}
\newtcolorbox[no counter]{definitions}[2][]{%
width=12cm,
center,
colback=lyellow!2!white,
colframe=yellow!3!lyellow,
coltitle=bule,
fonttitle=\bfseries,
rightrule=0pt,
leftrule=1pt,
bottomrule=2pt,
colbacktitle=lyellow,
theorem style=standard,
breakable,
arc=5pt,
enhanced,
drop fuzzy shadow southeast=black!20!white,
title=Definition. #2,#1}
\newtcolorbox[auto counter,number within=section]{corollarys}[2][]{%
colback=lyellow!2!white,
colframe=lyellow,
coltitle=bule,
fonttitle=\bfseries,
rightrule=0pt,
leftrule=1pt,
bottomrule=2pt,
colbacktitle=lyellow,
theorem style=standard,
breakable,
arc=0pt,
enhanced,
drop fuzzy shadow southeast=black!20!white,
title=Corollary~\thetcbcounter: #2,#1}
\newtcolorbox[auto counter,number within=section]{lemmas}[2][]{%
width=12cm,
center,
colback=lyellow!2!white,
colframe=lorange!30!sakura,
coltitle=bule,
fonttitle=\bfseries,
rightrule=0pt,
leftrule=1pt,
bottomrule=2pt,
colbacktitle=lorange!30!sakura,
theorem style=standard,
breakable,
arc=5pt,
enhanced,
drop fuzzy shadow southeast=black!20!white,
title=Lemma. #2,#1}
\newtcolorbox[auto counter,number within=section]{propositions}[2][]{%
width=12cm,
center,
colback=salmon!5,
colframe=salmon!90!black,
coltitle=white,
fonttitle=\bfseries,
rightrule=1pt,
leftrule=1pt,
bottomrule=2pt,
colbacktitle=salmon!90!black,
theorem style=standard,
breakable,
arc=5pt,
enhanced,
drop fuzzy shadow southeast=black!20!white,
title=Proposition. #2,#1}
\newtcolorbox[no counter]{egbox}[2][]{%
width=12cm,
center,
colback=black!5!white,
colframe=black!20!white,
coltitle=black,
fonttitle=\bfseries,
rightrule=1pt,
leftrule=1pt,
bottomrule=2pt,
colbacktitle=black!20!white,
theorem style=standard,
breakable,
arc=0pt,
enhanced,
drop fuzzy shadow southeast=black!20!white,
title=Example. #2,#1}

%\begin{figure}[H]
%\centering
%\includegraphics[center]{pic.png}
%\end{figure}
\geometry{left=3cm,right=3cm,top=2cm,bottom=2cm}
\tcbuselibrary{most}

\usepackage[linesnumbered,ruled,vlined]{algorithm2e}
\usepackage{algorithmic}

\SetKwProg{Fn}{function}{\string:}{}
\newcommand{\forcond}{$i=0$ \KwTo $n$}
\SetKwFunction{FRecurs}{FnRecursive}
\SetKwInput{KwCost}{Cost}

\usepackage{holtpolt}

%自定义设置
\renewcommand{\proofname}{Proof.}
\renewcommand{\contentsname}{ Content }
\newcommand{\image}[2]{
    \centering
    \includegraphics[width={#1}\textwidth]{#2}
}



\newcommand\keywords[1]{\vskip2ex\par\noindent\normalfont{\textbf{关键词}: #1}}
\newcommand{\ekeywords}[1]{\vskip2ex\par\noindent\normalfont{\bfseries Key Words: }#1}
\newcommand{\miku}{\textcolor{miku}}
\newcommand{\sakura}{\textcolor{sakura}}
\newcommand{\brown}{\textcolor{brow}}
\newcommand{\red}{\textcolor{red}}
\newcommand{\blue}{\textcolor{blue}}
\newcommand{\A}{\mathcal{A}}
\newcommand{\C}{\mathbb{C}}
\newcommand{\al}{\alpha}
\newcommand{\sa}{$\sigma$-algebra}
\newcommand{\Bsa}{Borel $\sigma$-algebra}
\newcommand{\F}{\mathcal{F}}
\newcommand{\N}{\mathcal{N}}
\newcommand{\M}{\mathcal{M}}
\newcommand{\m}{ $\mathcal{M}$ }
\newcommand{\B}{\mathcal{B}}
\newcommand{\myP}{\mathcal{P}}
\renewcommand{\bf}[1]{\textbf{#1}}

\newcommand{\myRom}[1]{\uppercase\expandafter{\romannumeral#1}}
\newcommand{\pl}{$ L^p(X) $}
\newcommand{\twol}{$ L^2(X) $}

\usepackage{booktabs}

\begin{document}
\hfill\vbox{\hbox{Numerical Analysis}\hbox{陈曦,HOME}\hbox{Summer, 2024}}

\begin{center}\Large
    \textbf{数值分析}\\{\normalsize\bf {数值积分和数值微分}}
\end{center}
\vskip 30pt
\small {参考书目:
\begin{itemize}
    \item Numerical Analysis(David Kincaid, Ward Cheney)
    \item Numerical Analysis(Timothy Sauer,2014)
    \item Approximation Theory and Approximation Practice(Trefethen,2013)
\end{itemize}}
\vskip 30pt

\section{数值微分}
最常见的数值微分是直接差分法。如果记$ D_+ $为前向差分算子,后向差分算子为$ D_- $,中心差分算子为$ D_0 $,即
\[
    D_+u(t) = \frac{u(t+h)-u(t)}{h},\quad D_-u(t) = \frac{u(t)-u(t-h)}{h},\quad D_0u(t) = \frac{u(t+h)-u(t-h)}{2h},
\]
则它们都可以用来近似$ u'(t) $,使用带余项的Taylor展开可以看到
\[
    D_+ u(t) = u'(t) + \frac{h}{2}u''(\xi_+) ,\quad D_- u(t) = u'(t) + \frac{h}{2}u''(\xi_-),\quad D_0 u(t) = u'(t) + \frac{h^2}{6}u'''(\xi_0),
\]
所以前两种单侧差分是一阶的,中心差分是二阶的。为了近似更高阶的导数,我们可以使用更高阶的差分算子,例如
\[
    D_+D_-u(t) = \frac{u(t+h)-2u(t)+u(t-h)}{h^2} = u''(t) + \frac{h^2}{12}u^{(4)}(\xi)
\]
是二阶中心差分,具有二阶精度;类似地,使用待定系数法,考虑
\[
    \sum_{k=-l}^r a_k u(t+kh) = u^{(n)}(t) + O(h^n),
\]
将左侧使用Taylor展开,令两侧各阶系数相等可以求出$ a_k $的值。另外,给出两个常用的单侧差分算子:
\[
    \frac{3u(t)-4u(t+h)+u(t+2h)}{-2h} = u'(t) + O(h^2),\quad \frac{u(t-2h)-4u(t-h)+3u(t)}{2h} = u'(t) + O(h^2).
\]

\subsection{插值型微分}
借助多项式插值,我们可以得到一大类数值微分方法,这类方法称为插值型微分,其基本思想是先使用多项式对原函数进行插值,之后使用该插值多项式的导数来近似原函数的导数。当使用Lagrange插值时,回忆其满足
\begin{equation}
    f(x) = \sum_{i=0}^n f(x_i) \ell_i(x) + \frac{f^{(n+1)}(\xi_x)}{(n+1)!} \prod_{i=0}^n (x-x_i),
\end{equation}
其中$ \ell_i(x) $是Lagrange插值基函数$ \prod_{j\ne i} (x-x_j) / (x_i-x_j) $。对上式的两侧求导可得
\begin{equation}
    f'(x) = \sum_{i=0}^n f(x_i) \ell_i'(x) + \frac{f^{(n+1)}(\xi_x)}{(n+1)!} \frac{d}{dx}\prod_{i=0}^n (x-x_i) + \frac{1}{(n+1)!} \prod_{i=0}^n (x-x_i)\frac{d}{dx}f^{(n+1)}(\xi_x).
\end{equation}
考虑该近似在某插值节点$ x_k $上的值,则
\begin{equation}
    f'(x_k) = \sum_{i=0}^n f(x_i) \ell_i'(x_k) + \frac{f^{(n+1)}(\xi_{x_k})}{(n+1)!} \frac{d}{dx}\prod_{i=0}^n (x-x_i)|_{x=x_k},
\end{equation}
因为
\[
    \frac{d}{dx}\prod_{i=0}^n (x_k-x_i) = \sum_{i=0}^n \prod_{j\ne i} (x_k-x_j) = \prod_{j\ne k} (x_k-x_j),
\]
所以插值型微分的误差为
\begin{equation}
    f'(x_k) - \sum_{i=0}^n f(x_i) \ell_i'(x_k) = \frac{f^{(n+1)}(\xi_{x_k})}{(n+1)!} \prod_{j\ne k} (x_k-x_j).
\end{equation}

\subsection{Richardson外推法}
注意到数值微分的误差往往形如
\[
    C h^n + O(h^{n+1}),
\]
其中$ C $是与$ h $无关的常数,因此我们可以使用外推法来提高精度,具体来说,我们可以使用两个不同的步长$ h_1 $和$ h_2 $,得到两个近似值$ u_1 $和$ u_2 $,将他们进行线性组合以消去误差中的主要部分从而提高数值精度。一般地,对于一个相容的数值微分方法我们有
\begin{equation}
    Lu = \varphi(h) + a_1 h + a_2 h^2 + a_3 h^3 + a_4 h^4 + \cdots,
\end{equation}
其中$ L $是差分算子,$ \varphi(h) $是使用$ u $在某些节点处的值组合得到的数值微分,$ h $是步长,剩余的部分是误差项。为了消除一阶误差项,分别使用$ h $和$ h/2 $的步长得到两个近似值$ u_1 $和$ u_2 $,则
\[
    \begin{aligned}
        L u 
        &= \varphi(h) + a_1 h + a_2 h^2 + a_3 h^3 + a_4 h^4 + \cdots,\\
        &= \varphi(h/2) + a_1 h/2 + a_2 h^2/4 + a_3 h^3/8 + a_4 h^4/16 + \cdots,
    \end{aligned}
\]
将第二行乘二并减去第一行,可以得到
\[
    Lu = 2\varphi(h / 2) - \varphi(h) - \frac{a_2}{2}h^2 - \frac{3a_3}{4}h^3 + \cdots,
\]
因此$ 2\varphi(h / 2) - \varphi(h) $是一个具有二阶截断误差的更高精度的数值微分。

以上方法称为Richardson外推法,重复使用它可以提高数值微分的精度。$ M $步的Richardson外推法在每步分为两个阶段:
\begin{enumerate}
    \item 选取合适的步长$ h $并计算前$ M+1 $个数值近似值,并记为
    \begin{equation}
        D(n,0) = \varphi(h / 2^n),\quad n=0,1,\cdots,M.
    \end{equation}
    \item 逐项消去高阶误差项,使用以下递推公式计算$ D(n,k) $:
    \begin{equation}
        D(n,k) = \frac{2^k}{2^k-1}D(n,k-1) - \frac{1}{2^k-1}D(n-1,k-1),
    \end{equation}
    其中$ n=k,\cdots,M $,$ k=1,2,\cdots,n $。
\end{enumerate}
使用数学归纳法可知按照以上方法得到的$ D(n,k) $具有$ k+1 $阶截断误差。

当使用中心差分时,误差中不含有奇数次的项,因此我们可以将第二个阶段中的递推关系改为
\begin{equation}
    D(n,k) = \frac{4^k}{4^k-1}D(n,k-1) - \frac{1}{4^k-1}D(n-1,k-1),
\end{equation}
此时$ D(n,k) $的截断误差阶数为$ 2(k+1) $。

通常Richardson外推法的计算借助于以下的表格:
\begin{center}
    \begin{tabular}{cccccc}
        \toprule
        $ D(0,0) $ & & & & & \\
        $ D(1,0) $ & $ D(1,1) $ & & & & \\
        $ D(2,0) $ & $ D(2,1) $ & $ D(2,2) $ & & & \\
        $ \vdots $ & $ \vdots $ & $ \vdots $ & $ \ddots $ & & \\
        $ D(M,0) $ & $ D(M,1) $ & $ D(M,2) $ & $ \cdots $ & $ D(M,M) $ & \\
        \bottomrule
    \end{tabular}
\end{center}
每次计算$ D(n,k) $时需要使用$ D(n,k-1) $和$ D(n-1,k-1) $,因此可以从左上角开始逐列计算。

\section{数值积分}
数值积分可以视作数值微分的逆运算,其目的是计算定积分
\[
    I = \int_a^b f(x)dx.
\]
最简单的积分方法包括矩形法、梯形法和Simpson法则,这些方法都是插值型积分方法的特例。所谓插值型积分方法是指先对被积函数进行多项式插值,之后对插值多项式进行积分。当使用Lagrange插值时,我们有
\[
    f(x) = \sum_{i=0}^n f(x_i) \ell_i(x) + \frac{f^{(n+1)}(\xi_x)}{(n+1)!} \prod_{i=0}^n (x-x_i),
\]
两侧同时积分可得
\[
    \int_a^b f(x)dx = \sum_{i=0}^n f(x_i) \int_a^b \ell_i(x)dx + \frac{f^{(n+1)}(\xi_x)}{(n+1)!} \int_a^b \prod_{i=0}^n (x-x_i)dx,
\]
记其中与$ f $无关的
\begin{equation}
    \int_a^b \ell_i(x)dx :=A_i,
\end{equation}
则$ f $的插值型数值积分可以表示为
\begin{equation}
    \sum_{i=0}^n A_i f(x_i) ,
\end{equation}
相应的数值误差为
\begin{equation}
    \int_a^b\frac{f^{(n+1)}(\xi_x)}{(n+1)!} \prod_{i=0}^n (x-x_i)dx,
\end{equation}
使用绝对值不等式可得
\begin{equation}
    \left\vert \int_a^b\frac{f^{(n+1)}(\xi_x)}{(n+1)!} \prod_{i=0}^n (x-x_i)dx \right\vert \leqslant  \frac{M}{(n+1)!} \int_a^b \prod_{i=0}^n \vert x-x_i \vert dx,
\end{equation}
其中$ M = \max_{x\in [a,b]}f^{(n+1)}(x) $。

与连续空间上的最佳逼近类似,我们希望最小化上述误差,为此需要寻找合适的插值节点$ x_i $,这一问题可以转化为如下最优化问题:
\begin{equation}
    \min_{x_0,\cdots ,x_n}\int_a^b \prod_{i=0}^n \vert x-x_i \vert dx,
\end{equation}
可以证明在$ [a,b] = [-1,1] $时,当且仅当$ x_0,\cdots ,x_n $是第二类Chebyshev点,即第二类Chebyshev多项式$ U_{n+1}(x) = \sin[(n+2)\arccos x] / \sin (\arccos x) $的零点时上式取到最小值$ 2^{-n} $,即
\begin{equation}
    \int_{-1}^1 \prod_{i=0}^n \vert x-x_i \vert dx = 2^{-n},
\end{equation}
相应的插值点,即第二类Chebyshev点为
\begin{equation}
    x_i = \cos\left(\frac{i+1}{n+2}\pi\right),\quad i=0,1,\cdots ,n.
\end{equation}
于是我们有如下误差估计:
\begin{equation}
    \left\vert \int_{-1}^1 f(x)dx - \sum_{i=0}^n A_i f(x_i) \right\vert \leqslant \frac{M}{2^{n}(n+1)!}.
\end{equation}

当区间为一般的$ [a,b] $时,使用坐标变换
\[
    x = \frac{b-a}{2}t + \frac{a+b}{2}
\]
可以将区间变换到$ [-1,1] $上,此时插值型积分的误差为
\begin{equation}
    \left\vert \int_a^b f(x)dx - \sum_{i=0}^n A_i f(x_i) \right\vert\leqslant \frac{M}{(n+1)!} \int_{-1}^1 \prod_{i=0}^n \frac{b-a}{2}\vert t-t_i \vert\cdot \frac{b-a}{2}dt \leqslant \frac{M}{2^{n}(n+1)!}\cdot (\frac{b-a}{2})^{n+2}.
\end{equation}

\subsection{Newton-Cotes公式}
如果不追求特别高的精度,等距节点的插值点是最简单的选择,相应的插值型积分方法称为Newton-Cotes插值。最简单的Newton-Cotes公式是矩形法、梯形法和Simpson法则,它们分别对应于$ n=0,1,2 $的情形。对于$ n=0 $的情形,我们有
\begin{equation}
    \int_a^b f(x)dx \approx f(a)(b-a) \text{ or } f(b)(b-a),
\end{equation}
也可以使用中点法,即
\begin{equation}
    \int_a^b f(x)dx \approx f(\frac{a+b}{2})(b-a).
\end{equation}
这种情况下我们只使用一个插值点$ x_0=a,b $或者$ (a+b) / 2 $。根据插值型积分的误差估计可知$ x_0=a,b $时方法的误差为
\[
    f'(\xi) \int_a^b (x-x_0) dx = \pm \frac{f'(\xi)}{2}(b-a)^2,
\]
不过当使用中点法时,由于正负误差恰好相互抵消,误差阶数上升为$ O((b-a)^3) $。

对于$ n=1 $的情形,我们有
\begin{equation}
    \int_a^b f(x)dx = \frac{f(a)+f(b)}{2}(b-a) + O((b-a)^3),
\end{equation}
注意到当$ x_0=a,x_1=b $时,Lagrange插值基函数为
\[
    \ell_0(x) = \frac{x-b}{a-b},\quad \ell_1(x) = \frac{x-a}{b-a},
\]
计算$ A_0,A_1 $可得
\[
    A_0 = \int_a^b \ell_0(x)dx = \frac{b-a}{2},\quad A_1 = \int_a^b \ell_1(x)dx = \frac{b-a}{2},
\]
所以我们有
\[
    \int_a^b f(x)dx \approx A_0 f(x_0) + A_1 f(x_1) =  \frac{f(a)+f(b)}{2}(b-a),
\]
由此可见梯形法则是$ n=1 $的Newton-Cotes公式。根据之前的分析可知
\[
    \int_a^b f(x)dx - \frac{f(a)+f(b)}{2}(b-a) = \int_a^b \frac{f''(\xi_x)}{2} (x-a)(x-b) dx = \frac{f''(\xi)}{2} \int_a^b (x-a)(x-b) dx,
\]
其中使用到了积分中值定理,于是该方法的误差为
\[
    \int_a^b f(x)dx - \frac{f(a)+f(b)}{2}(b-a) = -\frac{1}{12}f''(\xi)(b-a)^3.
\]

对于$ n=2 $的情形,Lagrange基函数变为
\[
    \ell_0(x) = \frac{(x-x_1)(x-x_2)}{(x_0-x_1)(x_0-x_2)},\quad \ell_1(x) = \frac{(x-x_0)(x-x_2)}{(x_1-x_0)(x_1-x_2)},\quad \ell_2(x) = \frac{(x-x_0)(x-x_1)}{(x_2-x_0)(x_2-x_1)},
\]
计算$ A_0,A_1,A_2 $可得
\[
    A_0 = \int_a^b \ell_0(x)dx = \frac{b-a}{6},\quad A_1 = \int_a^b \ell_1(x)dx = \frac{2(b-a)}{3},\quad A_2 = \int_a^b \ell_2(x)dx = \frac{b-a}{6},  
\]
由此可知$ n=2 $的Newton-Cotes公式为
\begin{equation}
    \int_a^b f(x)dx \approx A_0 f(x_0) + A_1 f(x_1) + A_2 f(x_2) = \frac{b-a}{6}\left[ f(a) + 4f\left(\frac{a+b}{2}\right) + f(b) \right],
\end{equation}
此即Simpson法则。与之前类似地借助积分中值定理可知该方法的误差为
\[
    \int_a^b f(x)dx - \frac{b-a}{6}\left[ f(a) + 4f\left(\frac{a+b}{2}\right) + f(b) \right] = -\frac{1}{90}f^{(4)}(\xi)(\frac{b-a}{2})^5.
\]

通常为了提高精度,我们可以将区间$ [a,b] $等分为$ n $个子区间,然后在每个子区间上使用Newton-Cotes公式,这种方法称为复合Newton-Cotes公式。当使用等距划分$ a = x_0< x_1<\cdots<x_{n-1} <x_n = b $时,复合梯形法则为
\begin{equation}
    \int_a^b f(x)dx = \sum_{i=1}^n \int_{x_{i-1}}^{x_i} f(x)dx \approx \frac{h}{2}\left[ f(x_0) + 2\sum_{i=1}^{n-1}f(x_i) + f(x_n) \right],
\end{equation}
其中$ h = (b-a)/n $,数值误差为
\[
    -\frac{1}{12} (b-a)h^2 f''(\xi).  
\]
复合Simpson法则为
\begin{equation}
    \int_a^b f(x)dx = \sum_{i=1}^n \int_{x_{i-1}}^{x_i} f(x)dx \approx \frac{h}{3}\left[ f(x_0) + 4\sum_{i=1}^{n / 2}f(x_{2i -1}) + 2\sum_{i=2}^{n / 2}f(x_{2i-2}) + f(x_n) \right],
\end{equation}
数值误差为
\[
    -\frac{1}{180} (b-a)h^4 f^{(4)}(\xi).
\]

一般的Newton-Cotes公式可以借助待定系数法来构造,由于插值节点已经确定,我们只需要计算$ A_i $的值。通常我们需要选取$ A_i $以使得相应的Newton-Cotes公式具有尽可能高的精度,一般$ n+1 $个节点的Newton-Cotes公式最高可以达到$ n $阶代数精度,即对于任意阶数不超过$ n $的多项式$ p(x) $,有
\[
    \int_a^b p(x)dx = \sum_{i=0}^n A_i p(x_i),
\]
即数值积分是准确的。要求$ n+1 $个节点的Newton-Cotes公式具有$ n $阶代数精度可以确定$ A_i $的值。要令$ n+1 $个节点的Newton-Cotes公式具有$ n $阶代数精度,只需
\begin{equation}
    \int_a^b x^k dx = \sum_{i=0}^n A_i x_i^k,\quad k=0,1,\cdots ,n,
\end{equation}
写成矩阵形式即
\[
    \begin{pmatrix}
        1 & 1 & \cdots & 1\\
        x_0 & x_1 & \cdots & x_n\\
        \vdots & \vdots & \ddots & \vdots\\
        x_0^n & x_1^n & \cdots & x_n^n
    \end{pmatrix}
    \begin{pmatrix}
        A_0\\
        A_1\\
        \vdots\\
        A_n
    \end{pmatrix}
    =
    \begin{pmatrix}
        b-a\\
        \frac{b^2-a^2}{2}\\
        \vdots\\
        \frac{b^{n+1}-a^{n+1}}{n+1}
    \end{pmatrix},
\]
注意到系数矩阵是Vandermonde矩阵且使用的是等距节点,因此该线性方程组是非奇异的,方程有界,解该线性方程组即可得到$ A_i $的值。然而该方法的缺点是当$ n $较大时,Vandermonde矩阵的条件数会变得很大,因此数值稳定性会变得很差。另一种方法是不再考虑$ x^k $,而是换用$ (x-x_0)\cdots (x-x_k) $,这种情况下有
\begin{equation}
    \int_a^b \prod_{j=0}^k (x-x_j)dx = \sum_{i=k+1}^n A_i \prod_{j=0}^k (x_i-x_j),
\end{equation}
于是矩阵形式变为
\[
    \begin{pmatrix} 
         1  & 1 & 1 & \cdots & 1\\
            & (x_1-x_0) & (x_2-x_0) & \cdots & (x_n-x_0)\\
            &  & (x_2-x_0)(x_2-x_1) & \cdots & (x_n-x_0)(x_n-x_1)\\
            &  &  & \ddots & \vdots\\
            &  &  &  & \prod_{j=0}^n (x_n-x_j)
    \end{pmatrix} 
    \begin{pmatrix}
        A_0\\
        A_1\\
        \vdots\\
        A_n
    \end{pmatrix}
    =
    \begin{pmatrix}
        b-a\\
        \int_a^b (x-x_0)dx\\
        \vdots\\
        \int_a^b \prod_{j=0}^n (x-x_j)dx
    \end{pmatrix},
\]
这样系数矩阵变为上三角矩阵,因此可以使用回代法求解。

更一般地,我们有广义Newton-Cotes公式,这类方法使用等距节点插值计算加权积分
\begin{equation}
    \int_a^b f(x)w(x)dx \approx \sum_{i=0}^n A_i f(x_i),
\end{equation}
其中$ w(x)>0 $是权函数,$ A_i $是待定系数。

\subsection{Gauss积分公式}
通过仔细选取插值节点,我们可以将插值型积分的精度提高到$ 2n+1 $阶,相应的积分方法称为Gauss积分。考虑一般的广义插值问题:
\begin{equation}
    \int_a^b f(x)w(x)dx = \sum_{i=0}^n A_i f(x_i),
\end{equation}
Gauss积分使用插值节点$ x_0,\cdots x_n $是关于$ w $的$ n+1 $阶广义正交多项式的零点。广义正交多项式的定义如下。
\begin{definition}
    如果$ n+1 $阶多项式$ Q $满足
    \begin{equation}
        \int_a^b P(x)Q(x)w(x)dx = 0,\quad \forall P\in \mathcal{P}_n,
    \end{equation}
    则称$ Q $是关于$ w $的$ n+1 $阶广义正交多项式。
\end{definition}
注意到对任意$ f\in \mathcal{P}_{2n+1} $都可以使用带余除法写为
\[
    f(x) = Q(x)g(x) + r(x),\quad \deg r < \deg Q = n+1,
\]
进而$ f(x_i) = r(x_i) $。又因为$ f $至多$ 2n+1 $阶,所以$ \deg g \leqslant n $,因此使用$ Q $的$ w$正交性有
\begin{equation}
    \int_a^b f(x)w(x)dx = \int_a^b Q(x)g(x)w(x)dx + \int_a^b r(x)w(x)dx = \sum_{i=0}^n A_i r(x_i) = \sum_{i=0}^n A_i f(x_i),
\end{equation}
因此该积分公式具有$ 2n+1 $阶代数精度。

在使用Gauss积分时首先需要确定相应的广义正交多项式,通常可以使用Gram-Schmidt正交化方法来构造,其中使用关于$ w $的广义内积:
\begin{equation}
    \langle f,g \rangle = \int_a^b f(x)g(x)w(x)dx.
\end{equation}
可以验证广义正交多项式的所有根都是$ (a,b) $内实的单根。

最后我们对Gauss积分的误差进行估计。我们先给出Gauss积分的一些性质。
\begin{lemma}
    Gauss积分公式中的系数$ A_i $都是正的,且满足
    \begin{equation}
        \sum_{i=0}^n A_i = \int_a^b w(x)dx.
    \end{equation}
\end{lemma}
\begin{proof}
    先说明$ A_i>0 $。令$ Q $是关于$ w $的$ n+1 $阶广义正交多项式,设它的根为$ x_0,\cdots ,x_n $,令$ P(x) = Q(x) / (x-x_i) $,其中$ x_i $是$ Q $的某一根,因此$ P $是$ n $阶多项式,因为Gauss积分具有$ 2n+1 $阶代数精度,所以对$ P^2 $使用Gauss积分有
    \[
        0 = \int_a^b P^2(x)w(x)dx = \sum_{j=0}^n A_j P^2(x_j) = A_i P^2(x_i),
    \]
    所以$ A_i>0 $。另外,对$ Q=1 $使用Gauss积分有
    \[
        \int_a^b w(x)dx = \sum_{i=0}^n A_i,
    \]
    证毕。
\end{proof}

根据Weierstrass定理,有界闭区间上的连续函数可以被多项式逼近,因此对于任意$ f\in C[a,b] $,给定任意$ \varepsilon>0 $,都存在$ P\in \mathcal{P} $使得
\[
    |f(x) - P(x)| < \varepsilon,\quad \forall x\in [a,b].
\]
使用这一事实我们可以证明使用Gauss积分计算$ f $的广义积分的误差随着插值节点个数$ n $的增大而收敛到零。

\begin{theorem}{\normalfont\bf{Gauss积分的收敛性}}
    如果$ f\in C[a,b] $,则对于任意$ \varepsilon>0 $,存在$ N $使得任意$ n>N $都有
    \begin{equation}
        \left\vert \int_a^b f(x)w(x)dx - \sum_{i=0}^n A_{ni} f(x_{ni}) \right\vert < \varepsilon,
    \end{equation}
    因此Gauss积分的误差随着插值节点个数$ n $的增大而收敛到零,其中下标$ ni $表示使用$ A_{ni} $和节点$ x_{ni} $与节点个数$ n $有关。
\end{theorem}
\begin{proof}
    选取$ \epsilon > 0 $使得
    \[
        2 \epsilon \int_a^b w(x)dx < \varepsilon,
    \]
    根据Weierstrass定理,存在$ P\in \mathcal{P} $使得
    \[
        |f(x) - P(x)| < \epsilon,\quad \forall x\in [a,b].
    \]
    使用三角不等式可知
    \[
        \begin{aligned}
            \left\vert \int_a^b f(x)w(x)dx - \sum_{i=0}^n A_{ni} f(x_{ni}) \right\vert &\leqslant \left\vert \int_a^b f(x)w(x)dx - \int_a^b P(x)w(x)dx \right\vert + \left\vert \int_a^b P(x)w(x)dx - \sum_{i=0}^n A_{ni} f(x_{ni}) \right\vert\\
            &\leqslant \int_a^b |f(x) - P(x)|w(x)dx + \left\vert \sum_{i=0}^n A_{ni} P(x_{ni}) - \sum_{i=0}^n A_{ni} f(x_{ni}) \right\vert\\
            &\leqslant \epsilon \int_a^b w(x)dx + \sum_{i=0}^n A_{ni} |P(x_{ni}) - f(x_{ni})|\\
            &\leqslant \epsilon \int_a^b w(x)dx + \sum_{i=0}^n A_{ni} \epsilon\\
            &\leqslant 2 \epsilon \int_a^b w(x)dx < \varepsilon,
        \end{aligned}
    \]
    其中我们令$ N $为任意大于$ \deg P $的整数,证毕。
\end{proof}

更进一步地,我们可以给出Gauss积分的具体误差公式。
\begin{theorem}{\normalfont\bf{$ n+1 $节点Gauss积分的误差公式}}
    如果$ f\in C^{2n+2}[a,b] $,则存在$ \xi\in (a,b) $使得
    \begin{equation}
        \int_a^b f(x)w(x)dx - \sum_{i=0}^n A_i f(x_i) = \frac{f^{(2n+2)}(\xi)}{(2n+2)!} \int_a^b \prod_{i=0}^n (x-x_i)dx.
    \end{equation}
\end{theorem}
\begin{proof}
    令$ P $为$ f $在Gauss插值节点处的二重Hermite插值多项式,即
    \[
        P(x_i) = f(x_i),\quad P'(x_i) = f'(x_i),\quad i=0,1,\cdots ,n,
    \]
    于是$ \deg P = 2n+1 $,插值误差为
    \[
        f(x) - P(x) = \frac{f^{(2n+2)}(\xi_x)}{(2n+2)!} \prod_{i=0}^n (x-x_i)^2,
    \]
    将上式关于$ w $积分并使用Gauss积分即可得到
    \[
        \int_a^b f(x)w(x)dx - \sum_{i=0}^n A_i f(x_i) = \frac{1}{(2n+2)!} \int_a^b f^{(2n+2)}(\xi_x)\prod_{i=0}^n (x-x_i)^2dx,
    \]
    使用积分中值定理可知上式右侧等于
    \[
        \frac{f^{(2n+2)}(\xi)}{(2n+2)!} \int_a^b \prod_{i=0}^n (x-x_i)dx,
    \]
    因此
    \[
        \int_a^b f(x)w(x)dx - \sum_{i=0}^n A_i f(x_i) = \frac{f^{(2n+2)}(\xi)}{(2n+2)!} \int_a^b \prod_{i=0}^n (x-x_i)dx,
    \]
    证毕。
\end{proof}

\subsection{Romberg积分}
当使用复合梯形法则时,类似于数值微分中的Richardson外推法,我们可以通过递推的方式提高数值积分的精度,在这个过程中,我们希望可以充分利用以往的计算结果以减少计算量(用内存换速度),相应的方法称为Romberg积分。

令计算步长为$ h = (b-a) / n $,则记复合梯形法则为
\begin{equation}
    T(n) = \frac{h}{2}\left[ f(x_0) + 2\sum_{i=1}^{n-1}f(x_i) + f(x_n) \right] = \frac{b-a}{2n} \left[ f(a) + 2\sum_{i=1}^{n-1}f(a+\frac{b-a}{n}i) + f(b) \right],
\end{equation}
注意到当步长减半时,原本的节点仍然在被使用,因此无需重复计算这一部分极点处的值,更准确地说,我们有
\begin{equation}
    \begin{aligned}
        T(2n) 
        &= \frac{b-a}{4n} \left[ f(a) + 2\sum_{i=1}^{2n-1}f(a+\frac{b-a}{2n}i) + f(b) \right]\\
        &= \frac{1}{2}T(n) + \frac{b-a}{2n} \sum_{i=1}^{n-1}f(a+\frac{b-a}{2n}(2i-1)),
    \end{aligned}
\end{equation}
重复使用上式可得
\begin{equation}
    T(2^n) = \frac{1}{2}T(2^{n-1}) + \frac{b-a}{2^n} \sum_{i=1}^{2^n-1}f(a+\frac{b-a}{2^n}(2i-1)),
\end{equation}
于是使用这一递推关系可以较高效地计算出$ T(2^n) $,这一方法称为Romberg积分。

通常Romberg积分可以借助Richardson外推法提高精度,这一过程往往使用表格来计算。令表格的第一列为$ R(n,0)=T(2^n) $。注意到梯形法则具有二阶精度且误差中不含有奇数次项,因此Richardson外推过程中的递推关系为
\begin{equation}
    R(n,k) = \frac{4^k}{4^k-1}R(n,k-1) - \frac{1}{4^k-1}R(n-1,k-1),
\end{equation}
其中$ n=2^k,\cdots ,M $,$ k=1,2,\cdots ,n $。计算表格为
\begin{center}
    \begin{tabular}{cccccc}
        \toprule
        $ R(0,0) $ & & & & & \\
        $ R(1,0) $ & $ R(1,1) $ & & & & \\
        $ R(2,0) $ & $ R(2,1) $ & $ R(2,2) $ & & & \\
        $ \vdots $ & $ \vdots $ & $ \vdots $ & $ \ddots $ & & \\
        $ R(M,0) $ & $ R(M,1) $ & $ R(M,2) $ & $ \cdots $ & $ R(M,M) $ & \\
        \bottomrule
    \end{tabular}
\end{center}

\end{document}